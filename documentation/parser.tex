Os dois parsers implementados (recursivo e por tabela) são parsers recursivos. Isso significa que não utilizam de backtracking e portanto rodam em tempo linear (de acordo com o tamnho do input, a palavra sendo analisada). Isso é importante pois o tempo de compilação de programas deve ser o menor possível. O parsing é feito analisando as regras sintáticas da linguagem e um input (uma palavra), e é utilizado para determinar se essa palavra faz parte da linguagem. 

\section{Conjuntos Predict}
Cada regra sintática tem associada a si um conjunto de terminais chamado \textit{Predict}.

Seja $A$ um símbolo não-terminal, $S$ o símbolo não-terminal inicial, $\alpha$ e $\beta$ sequências de não-terminais e/ou terminais e $T$ o conjunto dos símbolos terminais,

\begin{equation*}
Predict(A\rightarrow \alpha) = 
\begin{cases}
First(\alpha) \cup Follow(A)  & \text{ if } A \Rightarrow^* \lambda  \\
First(\alpha) & \text{otherwise}
\end{cases}
\end{equation*}

\begin{equation*}
First( \alpha) = \{s \in T \mid \alpha \Rightarrow^* s\beta\}
\end{equation*}

\begin{equation*}
Follow(A) = \{s \in T \mid S \Rightarrow^* ...As...\}
\end{equation*}

O conjunto Predict de uma regra $\alpha$ basicamente nos diz quais símbolos terminais são candidatos a serem o próximo a aparecer se usarmos $\alpha$. Tanto no parser recursivo como o por tabela, a cada passo nós estamos analisando um símbolo não-terminal $A$ (começando pelo símbolo inicial) e um terminal $s$ (vindo do input). Para sabermos qual regra utilizar, basta olharmos as regras de $A$ e acharmos a que tem $s$ no seu conjunto Predict. Se $s$ não estiver no conjunto Predict de nenhuma regra de $A$, o input não é uma palavra da linguagem, e um erro deve ser gerado. Como nossa gramática é LL1 (ignorando o caso do else), isso significa que não há interseção nos conjuntos Predict de regras de um mesmo símbolo não-terminal. Ou seja, para cada não-terminal $A$ e terminal $s$, no máximo uma regra de $A$ terá $s$ no seu Predict, permitindo que o parsing seja realizado da maneira descrita nesse parágrafo com sucesso. Os conjuntos Predict de todas as regras da nossa gramática podem ser vistos na Tabela \ref{tab:predict}.

\begin{center}
\begin{longtable}[H]{|C{5cm} | C{5cm} | C{5cm} |}
\caption{Predict}\label{tab:predict}\\
\hline
\textbf{Nonterminal} & \textbf{Rule} & \textbf{Predict} \\
\hline
PROGRAM &`program' `id' `;' DECL BLOCK `.' & \{`program'\} \\
\hline
DECL &CONSTS USERTYPES VARS SUBPROGRAMS &\{`const', `type', `var', `begin', `func', `proc'\} \\
\hline
\multirow{2}{*}{CONSTS} & `' & \{`var', `proc', `func', `type', `begin'\} \\ \cline{2-3}
&`const' LISTCONST & \{`const'\} \\
\hline
LISTCONST & CONSTDECL LISTCONSTPRIME & \{`id'\}\\
\hline
\multirow{2}{*}{LISTCONSTPRIME} & `' & \{`var', `proc', `func', `type', `begin'\} \\ \cline{2-3}
& LISTCONST & \{`id'\}\\
\hline
CONSTDECL & `id' `=' EXPR `;' & \{`id'\}\\
\hline
\multirow{2}{*}{TYPES} & `id' TYPESPRIME & \{`id'\} \\ \cline{2-3}
& PRIMTYPES & \{`int', `real', `bool', `char', `string', `array', `set', `(', `record'\}\\
\hline
\multirow{2}{*}{TYPESPRIME} & `' & \{`id', `;', \} \\ \cline{2-3}
& `..' SUBRANGETYPE & \{`..'\}\\
\hline
\multirow{9}{*}{PRIMTYPES} & `int' TYPESPRIME & \{`int' \} \\ \cline{2-3}
& `real' & \{`real'\}  \\ \cline{2-3}
& `bool' & \{`bool'\} \\ \cline{2-3}
& `char' & \{`char'\} \\ \cline{2-3}
& `string' & \{`string'\} \\ \cline{2-3}
& ARRAYTYPE & \{`array'\} \\ \cline{2-3}
& SETTYPE & \{`set'\} \\ \cline{2-3}
& ENUMTYPE & \{`('\} \\ \cline{2-3}
& RECORDTYPE & \{`record'\} \\ \cline{2-3}
\hline
ARRAYTYPE & `array' `[' SUBRANGELIST `]' `of' TYPES & \{`array'\} \\
\hline
\multirow{3}{*}{SUBRANGELIST} & `id' SUBRANGEPRIME & \{`id'\} \\ \cline{2-3}
& `int' '..' SUBRANGETYPE SUBRANGELISTPRIME & \{`int'\} \\ \cline{2-3}
& `char' '..' SUBRANGETYPE SUBRANGELISTPRIME & \{`char'\} \\
\hline
\multirow{2}{*}{SUBRANGEPRIME} & SUBRANGELISTPRIME & \{`]', `,'\} \\ \cline{2-3}
& `..' SUBRANGETYPE SUBRANGELISTPRIME & \{`..'\}\\
\hline
\multirow{2}{*}{SUBRANGELISTPRIME} & `' & \{`]'\} \\ \cline{2-3}
& `,' SUBRANGELIST & \{`,'\}\\
\hline
\multirow{3}{*}{SUBRANGETYPE} & `id' & \{`id'\} \\ \cline{2-3}
& `int' & \{`int'\} \\ \cline{2-3}
& `char' & \{`char'\}\\
\hline
SETTYPE & `set' `of' TYPES & \{`set'\}\\
\hline
ENUMTYPE & `(' IDLIST `)' & \{`('\} \\
\hline
RECORDTYPE & `record' VARLISTLIST `end' & \{`record'\} \\
\hline
\multirow{2}{*}{USERTYPES} & `' & \{`var', `begin', `proc', `func'\} \\ \cline{2-3}
& `type' LISTUSERTYPES & \{`type'\}\\
\hline
LISTUSERTYPES & USERTYPE LISTUSERTYPESPRIME & \{`id'\} \\
\hline
\multirow{2}{*}{LISTUSERTYPESPRIME} & `' & \{`var', `begin', `proc', `func'\} \\ \cline{2-3}
& LISTUSERTYPES & \{`id'\}\\
\hline
USERTYPE & `id' `=' TYPES `;' & \{`id'\}\\
\hline
\multirow{2}{*}{VARS} & `' & \{`begin', `proc', `func'\} \\ \cline{2-3}
& `var' VARLISTLIST & \{`var'\}\\
\hline
VARLISTLIST & VARLIST VARLISTLISTPRIME & \{`id', `int', `real', `bool', `char', `string', `array', `set', `(', `record'\}\\
\hline
\multirow{2}{*}{VARLISTLISTPRIME} & `' & \{`end', `begin', `proc', `func'\} \\ \cline{2-3}
& VARLISTLIST & \{`id', `int', `real', `bool', `char', `string', `array', `set', `(', `record'\}\\
\hline
VARLIST & TYPES IDLIST `;' & \{`id', `int', `real', `bool', `char', `string', `array', `set', `(', `record'\} \\
\hline
IDLIST & `id' IDATTR IDLISTPRIME & \{`id'\} \\
\hline
\multirow{2}{*}{IDLISTPRIME} & `' & \{`;', `)'\} \\ \cline{2-3}
& `,' IDLIST & \{`,'\}\\
\hline
\multirow{2}{*}{IDATTR} & `' & \{`;', `,', `)'\} \\ \cline{2-3}
& `=' EXPR & \{`='\}\\
\hline
\multirow{2}{*}{VARIABLE} & `$=>$' `id' VARIABLEPRIME & \{`$=>$'\} \\ \cline{2-3}
& `[' EXPRLISTPLUS `]' VARIABLEPRIME & \{`['\}\\
\hline
\multirow{2}{*}{VARIABLEPRIME} & `' `id' VARIABLEPRIME & \{`id', `;', `]', `of', `,', `)', `end', `begin', `label', `exitwhen', `return', `:=', `if', `else', `loop', `case', `goto', `for', `to', `step', `do', `$||$', `\&\&', `+', `-', `*', `/', `\%', `$==$', `$!=$', `<', `$<=$', `<', `$>=$', `write', `writeln', `read', `readln'\} \\ \cline{2-3}
& VARIABLE & \{`[', `$=>$'\}\\
\hline
BLOCK & `begin' STMTS `end' & \{`begin'\} \\
\hline
STMTS & STMT STMTLISTPRIME & \{`id', `;', `end', `begin', `label', `exitwhen', `return', `if', `loop', `case', `goto', `for', `write', `writeln', `read', `readln'\} \\
\hline
\multirow{2}{*}{STMTLISTPRIME} & `' & \{`end'\} \\ \cline{2-3}
& `;' STMTS & \{`;'\}\\
\hline
\multirow{15}{*}{STMT} & `' & \{`;', `end', `else'\} \\ \cline{2-3}
& `label' STMT & \{`label'\}\\ \cline{2-3}
& `BLOCK & \{`begin'\}\\ \cline{2-3}
& WRITESTMT & \{`write'\}\\ \cline{2-3}
& WRITELNSTMT & \{`writeln'\}\\ \cline{2-3}
& READSTMT & \{`read'\}\\ \cline{2-3}
& READLNSTMT & \{`readln'\}\\ \cline{2-3}
& LOOPBLOCK & \{`loop'\}\\ \cline{2-3}
& IFBLOCK & \{`if'\}\\ \cline{2-3}
& FORBLOCK & \{`for'\}\\ \cline{2-3}
& CASEBLOCK & \{`case'\}\\ \cline{2-3}
& GOTOSTMT & \{`goto'\}\\ \cline{2-3}
& 'id' STMTPRIME & \{`id'\}\\ \cline{2-3}
& EXITSTMT & \{`exitwhen'\}\\ \cline{2-3}
& RETURNSTMT & \{`return'\}\\
\hline
\multirow{2}{*}{STMTPRIME} & ATTRSTMT & \{`[', `:=', `-$>$ '\} \\ \cline{2-3}
& SUBPROGCALL & \{`('\} \\
\hline
SUBPROGCALL & `(' EXPRLIST `)' & \{`('\} \\
\hline
EXITSTMT & `exitwhen' EXPR & `exitwhen'\\
\hline
RETURNSTMT & `return' EXPR & `return'\\
\hline
\multirow{2}{*}{ATTRSTMT} & VARIABLE ':=' EXPR & \{`[', `-$>$'\}\\ \cline{2-3}
& ':=' EXPR & \{`:='\}\\
\hline
IFBLOCK & 'if' EXPR STMT ELSEBLOCK & \{`if'\}\\
\hline
\multirow{2}{*}{ELSEBLOCK} & `' & \{`;', `end'\}\\ \cline{2-3}
& `else' STMT & \{`else'\}\\
\hline
LOOPBLOCK & `loop' STMT & \{`loop'\}\\
\hline
CASEBLOCK & `case' EXPR 'of' CASELIST CASEBLOCKPRIME & \{`case'\}\\
\hline
\multirow{2}{*}{CASEBLOCKPRIME} & `end' & \{`end'\}\\ \cline{2-3}
& `else' STMT `end' & \{`else'\}\\
\hline
CASELIST & LITERALLIST `:' STMT `;' & \{`intliteral', `realliteral', `charliteral', `stringliteral', `subrangeliteral'\}\\
\hline
LITERALLIST & LITERAL LITERALLISTPRIME & \{`intliteral', `realliteral', `charliteral', `stringliteral', `subrangeliteral'\}  \\
\hline
\multirow{2}{*}{LITERALLISTPRIME} & `' & \{`:'\}  \\ \cline{2-3}
& `,' LITERALLIST & \{`,'\}\\
\hline
GOTOSTMT & `goto' `label' & \{`goto'\}\\
\hline
FORBLOCK & `for' `id' FORBLOCKPRIME & \{`for'\}\\
\hline
\multirow{2}{*}{FORBLOCKPRIME} & VARIABLE `:=' EXPR `to' EXPR `step' EXPR `do' STMT & \{`[', `-$>$'\}\\ \cline{2-3}
& `:=' EXPR `to' EXPR `step' EXPR `do' STMT & \{`:='\}\\
\hline
EXPR & CONJ DISJ & \{`id', `(', `!', `intliteral', `realliteral', `charliteral', `stringliteral', `subrangeliteral'\}\\
\hline
\multirow{3}{*}{FINAL\_TERM} & `id' FINAL\_TERMPRIME & \{`id'\}\\ \cline{2-3}
& LITERAL & \{`intliteral', `realliteral', `charliteral', `stringliteral', `subrangeliteral'\}\\ \cline{2-3}
& `(' EXPR `)' & \{`('\}\\
\hline
\multirow{3}{*}{FINAL\_TERMPRIME} & VARIABLE & \{`[', `-$>$'\}\\ \cline{2-3}
& `' & \{`id', `;', `]', `of', `,', `)', `end', `begin', `label', `exitwhen', `return', `if', `else', `loop', `case', `goto', `for', `to', `step', `do', `$\mid \mid$', `\&\&', `+', `-', `*', `/', `\%', `==', `!=', `$<$', `$<$=', `$>$', `=$>$', `write', `writeln', `read', `readln'\}\\ \cline{2-3}
& SUBPROGCALL & \{`('\}\\
\hline
\multirow{2}{*}{DISJ} & `' & \{`id', `;', `]', `of', `,', `)', `end', `begin', `label', `exitwhen', `return', `if', `else', `loop', `case', `goto', `for', `to', `step', `do', `write', `writeln', `read', `readln'\}\\ \cline{2-3}
& `$\mid \mid$' CONJ & \{`$\mid \mid$'\}\\
\hline
CONJ & COMP CONJPRIME & \{`id', `(', `!', `intliteral', `realliteral', `charliteral', `stringliteral', `subrangeliteral'\}\\
\hline
\multirow{2}{*}{CONJPRIME} & `' & \{`id', `;', `]', `of', `,', `)', `end', `begin', `label', `exitwhen', `return', `if', `else', `loop', `case', `goto', `for', `to', `step', `do', `$\mid \mid$', `write', `writeln', `read', `readln'\} \\
\cline{2-3}
& `\&\&' COMP & \{`\&\&'\} \\
\hline
COMP & RELATIONAL COMPPRIME & \{`id', `(', `!', `intliteral', `realliteral', `charliteral', `stringliteral', `subrangeliteral'\} \\
\hline
RELATIONAL & SUM RELATIONALPRIME & \{`id', `(', `!', `intliteral', `realliteral', `charliteral', `stringliteral', `subrangeliteral'\} \\
\hline
\multirow{2}{*}{RELATIONALPRIME} & `' & \{`id', `;', `]', `of', `,', `)', `end', `begin', `label', `exitwhen', `return', `if', `else', `loop', `case', `goto', `for', `to', `step', `do', `$\mid \mid$', `==', `!=', `\&\&', `write', `writeln', `read', `readln'\} \\
\cline{2-3}
& RELATIONAL\_OP SUM & \{`$<$', `$<$=', `$>$', `=$>$'\} \\
\hline
\multirow{2}{*}{COMPPRIME} & `' & \{`id', `;', `]', `of', `,', `)', `end', `begin', `label', `exitwhen', `return', `if', `else', `loop', `case', `goto', `for', `to', `step', `do', `$\mid \mid$', `\&\&', `write', `writeln', `read', `readln'\} \\
\cline{2-3}
& EQUALITY\_OP RELATIONAL & \{`==', `!='\} \\
\hline
SUM & NEG SUMPRIME & \{`id', `(', `!', `intliteral', `realliteral', `charliteral', `stringliteral', `subrangeliteral'\} \\
\hline
\multirow{2}{*}{SUMPRIME} & '' & \{`id', `;', `]', `of', `,', `)', `end', `begin', `label', `exitwhen', `return', `if', `else', `loop', `case', `goto', `for', `to', `step', `do', `$\mid \mid$', `==', `!=' `\&\&', `$<$', `$<$=', `$>$', `=$>$', `write', `writeln', `read', `readln'\} \\
\cline{2-3}
& ADD\_OP NEG SUMPRIME & \{`+', `-'\} \\
\hline
\multirow{2}{*}{NEG} & MUL & \{`id', `(', `intliteral', `realliteral', `charliteral', `stringliteral', `subrangeliteral'\} \\
\cline{2-3}
& '!' MUL & \{`!'\} \\
\hline
MUL & FINAL\_TERM MULPRIME & \{`id', `(', `intliteral', `realliteral', `charliteral', `stringliteral', `subrangeliteral'\}\\
\hline
\multirow{2}{*}{MULPRIME} & `' & \{`id', `;', `]', `of', `,', `)', `end', `begin', `label', `exitwhen', `return', `if', `else', `loop', `case', `goto', `for', `to', `step', `do', `$||$', `$+$', `$-$', `==', `!=', `$<$', `$<=$', `$>$', `$>=$', `\&\&', `write', `writeln', `read', `readln'\} \\
\cline{2-3}
& MUL\_OP FINAL\_TERM MULPRIME & \{`$*$', `$/$', `$\%$'\} \\
\hline
\multirow{2}{*}{ADD\_OP} & `$+$' & \{`$+$'\} \\
\cline{2-3}
& `$-$' & \{`$-$'\} \\
\cline{2-3}
\hline
\multirow{3}{*}{MUL\_OP} & `$*$' & \{`$*$'\} \\
\cline{2-3}
& `$/$' & \{`$/$'\} \\
\cline{2-3}
& `$\%$' & \{`$\%$'\} \\
\cline{2-3}
\hline
\multirow{2}{*}{EQUALITY\_OP} & `$==$' & \{`$==$'\} \\
\cline{2-3}
& `$!=$' & \{`$!=$'\} \\
\cline{2-3}
\hline
\multirow{4}{*}{RELATIONAL\_OP} & `$<$' & \{`$<$'\} \\
\cline{2-3}
& `$<=$' & \{`$<=$'\} \\
\cline{2-3}
& `$>$' & \{`$>$'\} \\
\cline{2-3}
& `$>=$' & \{`$>=$'\} \\
\hline
\multirow{5}{*}{LITERAL} & `intliteral' & \{`intliteral'\} \\
\cline{2-3}
& `realliteral' & \{`realliteral'\} \\
\cline{2-3}
& `charliteral' & \{`charliteral'\} \\
\cline{2-3}
& `stringliteral' & \{`stringliteral'\} \\
\cline{2-3}
& `subrangeliteral' & \{`subrangeliteral'\} \\
\hline
\multirow{2}{*}{EXPRLIST} & `' & \{`)'\} \\
\cline{2-3}
& EXPRLISTPLUS & \{`id', `(', `!', `intliteral', `realliteral', `charliteral', `stringliteral', `subrangeliteral'\} \\
\hline
EXPRLISTPLUS & EXPR EXPRLISTPLUSPRIME & \{`id', `(', `!', `intliteral', `realliteral', `charliteral', `stringliteral', `subrangeliteral'\} \\
\hline
\multirow{2}{*}{EXPRLISTPLUSPRIME} & `' & \{`]', `)'\} \\
\cline{2-3}
&  `,' EXPRLISTPLUS & \{`,'\} \\
\hline
\multirow{3}{*}{SUBPROGRAMS} & `' & \{`begin'\} \\
\cline{2-3}
& PROCEDURE SUBPROGRAMSPRIME & \{`proc'\} \\
\cline{2-3}
& FUNCTION SUBPROGRAMSPRIME & \{`func'\} \\
\hline
\multirow{2}{*}{SUBPROGRAMSPRIME} & `' & \{`begin'\} \\
\cline{2-3}
& ';' SUBPROGRAMS & \{`;'\} \\
\hline
PROCEDURE & 'proc' 'id' '(' PARAM ')' ';' DECL BLOCK & \{`proc'\} \\
\hline
FUNCTION & 'func' TYPES 'id' '(' PARAM ')' ';' DECL BLOCK & \{`func'\} \\
\hline
\multirow{2}{*}{PARAM} & `' & \{`)'\} \\
\cline{2-3}
& PARAMLISTLIST & \{`id', `int', `real', `bool', `char', `string', `array', `set', `(', `record'\} \\
\hline
PARAMLISTLIST & PARAMLIST PARAMLISTLISTPRIME & \{`id', `int', `real', `bool', `char', `string', `array', `set', `(', `record'\} \\
\hline
\multirow{2}{*}{PARAMLISTLISTPRIME} & `' & \{`)'\} \\
\cline{2-3}
& `;' PARAMLISTLIST & \{`;'\} \\
\hline
\multirow{2}{*}{PARAMLIST} & `ref' TYPES IDLIST & \{`)'\} \\
\cline{2-3}
& TYPES IDLIST & \{`id', `int', `real', `bool', `char', `string', `array', `set', `(', `record'\} \\
\hline
WRITESTMT & `write' `(' EXPR `)' & \{`write'\} \\
\hline
WRITELNSTMT & `writeln' `(' EXPR `)' & \{`writeln'\} \\
\hline
READSTMT & `read' `(' `id' VARIABLEPRIME `)' &  \{`read'\} \\
\hline
READLNSTMT & `readln' `(' `id' VARIABLEPRIME `)' & \{`readln'\} \\
\hline
\end{longtable}
\end{center}

\section{Parser preditivo recursivo}
No parser recursivo, é utilizada a pilha de execução do programa para fazer o parsing top-down. Há uma funcão para cada não-terminal e o programa é inicilizado chamando a função do não-terminal inicial. Na função de cada não-terminal $A$, fazemos um switch/case utilizando o próximo símbolo terminal $t$ainda não \textit{matched} do input, onde cada case é a regra tal que $t$ está no Predict dessa regra: para cada não-terminal $B$ da regra, chamamos a função de $B$, e para cada terminal $s$ da regra, chamamos um \textit{matching} de $s$ e $t$ (que gera um erro se $s$ e $t$ forem diferentes e avança no input para o próximo símbolo). O caso default é onde $t$ não está no Predict de nenhuma regra de $A$, e gera um erro.  

\section{Parser preditivo por tabela}
No parser por tabela, utilizamos uma pilha explícita e uma tabela. No nosso código, a tabela é um std::map onde a chave é um par de um símbolo não-terminal e um símbolo terminal e o valor é um vetor de símbolos terminais e não-terminais (uma regra). Essa tabela é construída utilizando os conjuntos Predict; há uma entrada na tabela para a chave ($A$,$s$) se $s$ está no Predict de alguma regra de $A$, e o valor associado a essa chave é justamente essa regra. O programa é inicilizado empilhando o símbolo não-terminal inicial, e então fica em um loop enquanto a pilha não estiver vazia. A cada passo desse loop, olhamos o símbolo no topo da pilha. Se for um símbolo terminal $s$, fazemos o \textit{matching} com o próximo símbolo terminal $t$ ainda não \textit{matched} do input (que gera um erro se $s$ e $t$ forem diferentes e avança no input para o próximo símbolo). Se for um símbolo não-terminal $A$, olhamos na tabela a chave {$A$,$t$} e empilhamos a regra (o vetor) correspondente. Caso não haja um vetor para essa chave, um erro é gerado.

\section{Recuperação de erro}
O compilador não deve parar sua execução ao encontrar um erro sintático, mas sim notificar o erro e continuar o parsing. Desse modo, vários erros podem ser reportados com uma só compilação. Para isso, ao encontrar um erro, é preciso utilizar de alguma técnica de recuperação de erro para que o parsing possa continuar. No nosso caso, utilizamos diferentes técnicas de recuperação de erro. 

No parser recursivo, quando há um erro de \textit{matching} ou um erro de símbolo não esperado (quando não há regra cujo Predict contém o símbolo), esse erro é notificado e então avançamos no input para o próximo símbolo. Isso é uma boa técnica quando erros de digitação são frequentes, mas resulta em muitos outros erros se há uma palavra faltando no input. 

Já no parser por tabela, o erro de símbolo não esperado é recuperado de uma forma diferente: desempilhamos a pilha até encontrarmos um $;$ (ou até ela ficar vazia) e avançamos no input até encontrarmos um $;$ (ou até ele acabar). A intuição é que estamos continuando o parsing a partir do próximo $;$. O erro de \textit{matching} é recuperado da mesma forma que no recursivo.

\section{Saídas dos parsers}
Os parsers não tem saída quando não há erros, ou seja, quando o programa (o input) for uma palavra da linguagem. Para todos os nossos programas de exemplo (factorial.pie, quicksort.pie e mergesort.pie), os dois parsers não estão imprimindo nenhuma saída, o que significa que esses programas estão corretos em relação a linguagem.

Quando há um erro, ele é disponibilizado das seguintes formas:
\vspace{0.5cm}

\begin{verbatim}
ERROR: At row 48 column 24 
    Not expected symbol: ;
ERROR: At row 49 column 17 
    Not expected symbol: ID_TOKEN (with lexeme k)
ERROR: At row 49 column 19
    Expected: END_TOKEN
    Found: ATTR_TOKEN
\end{verbatim} 



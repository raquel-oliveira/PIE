
Criamos um back-end do compilador para a geração do código. 
A linguagem usada na criação do código intermediário é uma restrição da linguagem C que permite usar apenas os seguintes elementos:
\begin{itemize}
	\item Expressões aritméticas, lógicas, relacionais e chamadas de funções.
	\item Todos os tipos de dados da linguagem.
	\item Todas as declarações da linguagem.
	\item Apenas os seguintes comandos de C:
	\begin{itemize}
		\item Abertura e fechamento de blocos;
		\item Sequenciamento de comandos;
		\item Atribuição;
		\item Chamadas de funções/procedimentos (incluindo de entrada e saída);
		\item Rótulos (labels) e comando goto;
		\item Comandos return, break e exit;
		\item Comandos de seleção APENAS da forma:
		\begin{itemize}
			\item \begin{verbatim}
			if( condição ) goto l;
			\end{verbatim}
			\item \begin{verbatim}
		switch ( expressão ) {
		case valor : { … }
		…
		}
		\end{verbatim}
		\item Nenhum outro comando da linguagem deve ser usado, isto inclui comandos de iteração ou de
		seleção estruturados.
		\end{itemize}
	\end{itemize}
\end{itemize}

Ao compilar um arquivo $.pie$ (como descrito em \ref{ch:uso}), um código intermediário é criado na pasta $generated\_code$ com o mesmo nome do arquivo de input, porém com a extensão $.c$. Esse arquivo pode ser compilado através do compilador $gcc$, por exemplo:
\begin{verbatim}
gcc <nomedoarquivo>.c -o <nomedoexecutavel>
\end{verbatim}

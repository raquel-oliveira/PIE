Meta-operadores para definir as expressões regulares:
\begin{verbatim}
    [ ] : para enumerações associadas à -
    *   : repetição
    +   : para repetições de um vez ou mais
    ?   : de zero a uma vez
    .   : como um caractere ``joker'' exceto \n
    ^   : para o complementar de um  [ ]
    |   : para representar uma alternativa
\end{verbatim}

\section{id}
Identificadores podem começar apenas com letras, podem ter números e underline (``\_'') em sua estrutura.
A expressão regular que gera um identificador correto é:
 \begin{verbatim}
   id :  [a-zA-Z][a-zA-Z0-9_]*
\end{verbatim}

\section{label}
 \begin{verbatim}
   label : "@"[a-zA-Z0-9_]*
\end{verbatim}

\section{char}
\begin{verbatim}
   charliteral : \`[^']*\'
\end{verbatim}

\section{string}
\begin{verbatim}
   stringliteral : \"[^"\n]*\"
\end{verbatim}

\section{int}
\begin{verbatim}
   intliteral :  (("-"|"+")?[0-9]+)
\end{verbatim}

\section{bool}
\begin{verbatim}
   boolliteral :  "true"|"false"
\end{verbatim}

\section{real}
\begin{verbatim}
   exponent : ([E|e]("+"|"-")?({DIGIT}+))
   real : ([0-9]*[.])?[0-9]+
   realexponent : ([0-9]*[.])?[0-9]+{exponent}?
   realliteral :  (("-"|"+")?{real}|("-"|"+")?{realexponent})
\end{verbatim}

\section{Precedência}
A ordem de precedência deve valer para os seguintes operadores (), [], \{\}, $*$, $/$, \%, !, +, -, $<$, $<=$, $>$, $>=$, $:=$, $=$, $==$, $!=$, $\&\&$, $||$. A ordem de precedência pode ser visualizada na Tabela \ref{table:ordem_precedencia}.

\begin{table*}[h]
\renewcommand{\arraystretch}{1.34}
\centering
\begin{tabular}{| c | c | c | c | c | c |}
\hline
\bfseries Operador & \bfseries Precedência  \\
\hline
() & maior \\ \hline
[] &  \\ \hline
\{\} &  \\ \hline
 *, /, \% & \\ \hline
  ! & \\ \hline
 +, - & \\ \hline
$<$, $<=$, $>$, $>=$ & \\ \hline
$==$, $!=$ & \\ \hline
\&\& & \\ \hline
$||$ & \\ \hline
$:=$, = & menor\\ \hline
\end{tabular}
\caption{Ordem de precedência para os operadores.}
\label{table:ordem_precedencia}
\end{table*}

% Pascal tem funções matemáticas na própria linguagem (vamos abortar ou trataremos como se fosse de uma biblioteca?)
\section{Regra para comentários}
    Comentários iniciam por ``\#'' e são eliminados no pré-processamento.
\begin{verbatim}
   linecomment : "#"((.)*)\n
\end{verbatim}

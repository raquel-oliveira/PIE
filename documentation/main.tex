\documentclass[12pt]{report}
\usepackage[utf8]{inputenc}
\usepackage[brazilian]{babel}
%\usepackage[protrusion=true,expansion=true]{microtype} % S2  %
\usepackage{listings}
\usepackage{fullpage} % changes the margin
\usepackage{pxfonts} % to bold in listings
\usepackage{xcolor}
\usepackage{fancyvrb}
\usepackage{float}
\usepackage{txfonts}
\let\iint\relax
\let\iiint\relax
\let\iiiint\relax
\let\idotsint\relax
\usepackage{amsmath}
\usepackage{comment}

\def\changemargin#1#2{\list{}{\rightmargin#2\leftmargin#1}\item[]}
\let\endchangemargin=\endlist

\usepackage{titlesec}
\usepackage{longtable}
\usepackage{multirow}
\usepackage{array}
\usepackage{indentfirst}

\newcolumntype{L}[1]{>{\raggedright\let\newline\\\arraybackslash\hspace{0pt}}m{#1}}
\newcolumntype{C}[1]{>{\centering\let\newline\\\arraybackslash\hspace{0pt}}m{#1}}
\newcolumntype{R}[1]{>{\raggedleft\let\newline\\\arraybackslash\hspace{0pt}}m{#1}}
\usepackage{verbatim}


%definition of language PIE
\lstdefinelanguage{pie}
{
    morekeywords={[1]
        program,
        proc,
        begin,
        end,
        func,
        const,
        type,
        var
    },
    morekeywords={[2]
        if,
        else,
        goto,
        of, %array, set, subrange
        for,
        to, %for
        do, %for
        step, %for
        in, %subrange, set
        loop,
        exitwhen, %loop
        case,
        write,
        writeln,
        read,
        readln,
        return,
        ref
    },
    morekeywords={[3]
        int,
        bool,
        real,
        char,
        string,
        array,
        set,
        record,
        enum,
        subrange
    },
    morekeywords={[4]
        true, false, nil
    },
    morekeywords={[5]
        id,
        label,
        intliteral,
        realiteral,
        charliteral,
        stringliteral
    },
    sensitive=true,
    morecomment=[l]{\#},
    morestring=[b]",
}


\definecolor{keywordcolor}{RGB}{0,0,0}
\definecolor{commentcolor}{RGB}{50,50,50}
\definecolor{opcolor}{RGB}{0,0,150}
\definecolor{literalcolor}{RGB}{0,0,150}
\definecolor{todefine}{RGB}{205,92,150}
\definecolor{typecolor}{RGB}{0,150,0}

\lstdefinestyle{pie}
{
    language=pie,
    basicstyle=\ttfamily,
    numbers=left,
    numberstyle=\tiny,
    keywordstyle=[1]\bfseries\color{keywordcolor},
    keywordstyle=[2]\bfseries\color{keywordcolor},
    keywordstyle=[3]\bfseries\color{keywordcolor},
    keywordstyle=[4]\bfseries\color{literalcolor},
    keywordstyle=[5]\bfseries\color{todefine},
    commentstyle=\color{commentcolor},
    showstringspaces=false
}

\lstset{style=pie}

%
 
\lstset{language=pie}

\usepackage{hyperref}
\hypersetup{
    colorlinks=true,
    linkcolor=black,
    urlcolor=black,
    linktoc=all
}

\usepackage{titlesec}

\titleformat{\chapter}
  {\Large\bfseries} % format
  {}                % label
  {0pt}             % sep
  {\huge}           % before-code



\title{DIM0661-PB3}
%\subtitle{Definição da linguagem e análise léxica}
%\subtitle{Analisadores Sintáticos Preditivos}
%\subtitle{Analisador Sintático Ascendente}
\author{Grupo 3}
\date{\today}

\begin{document}

\maketitle

\tableofcontents

\chapter{Introdução}
Este relatório apresenta o manual da linguagem de programação que está sendo desenvolvida na disciplina de Compiladores (DIM0661). A linguagem deve obedecer as seguintes restrições:

\begin{changemargin}{1cm}{1cm}
\begin{itemize}
    \item deve ser parecida com Pascal (Pascal-like);
    \item deve ser em inglês;
    \item não deve possuir \texttt{while} e nem \texttt{repeat until};
    \item deve possuir um loop geral e permitir uma saída do loop (\texttt{exitwhen});
    \item deve ter tipagem fraca.
\end{itemize}
\end{changemargin}

O nome escolhido para a linguagem foi PIE, um acrônimo para Pascal-like (\textbf{P}ascal-l\textbf{I}k\textbf{E}).

\chapter{Regras sintáticas}
As regras sintáticas da linguagem foram construídas utilizando uma gramática livre de contexto que utiliza o formalismo de Backus-Naur (BNF).

\section{Gramática}

\begin{footnotesize}
\begin{lstlisting}[frame=single, label={prog}, language=pie]
<prog> ::= program id `;' <decl> <block> `.'
\end{lstlisting}

\begin{lstlisting}[frame=single, label={decl}, language=pie]
<decl> ::= <consts> <usertypes> <vars> <subprograms>
\end{lstlisting}

\begin{lstlisting}[frame=single, label={consts}, language=pie]
<consts> ::= `'                    |
             const <listconst> `;'
\end{lstlisting}

\begin{lstlisting}[frame=single, label={listconst}, language=pie]
<listconst> ::=  <constdecl> <listconstprime>
\end{lstlisting}

\begin{lstlisting}[frame=single, label={listconstprime}, language=pie]
<listconstprime> ::=  `'  |
		    <listconst>
\end{lstlisting}

\begin{lstlisting}[frame=single, label={constdecl}, language=pie]
<constdecl> ::= id `=' <expr> `; '
\end{lstlisting}

\begin{lstlisting}[frame=single, label={types}, language=pie]
<types> ::= id <typesprime> |
           <primtypes>
\end{lstlisting}

\begin{lstlisting}[frame=single, label={typesprime}, language=pie]
<typesprime> ::= `'                  |
                 `..' <subrangetype>

\end{lstlisting}

\begin{lstlisting}[frame=single, label={primtypes}, language=pie]
<primtypes> ::= int <typesprime>    |
                real                |
                bool                |
                char   <typesprime> |
                string              |
                <arraytype>         |
                <settype>           |
                <enumtype>          |
                <recordtype>
\end{lstlisting}

\begin{lstlisting}[frame=single, label={arraytype}, language=pie]
<arraytype> ::= array `[' <subrangelist> `]' of <types>
\end{lstlisting}

\begin{lstlisting}[frame=single, label={subrangelist}, language=pie]
<subrangelist> ::= id <subrangeprime>  |
                   int `..' <subrangetype> <subrangelistprime> |
                   char `..' <subrangetype> <subrangelistprime>
\end{lstlisting}

\begin{lstlisting}[frame=single, label={subrangeprime}, language=pie]
<subrangeprime> ::= <subrangelistprime>  |
                   `..' <subrangetype> <subrangelistprime>
\end{lstlisting}


\begin{lstlisting}[frame=single, label={subrangelistprime}, language=pie]
<subrangelistprime> ::=  `' |
	                 `,' <subrangelist>
\end{lstlisting}
\begin{lstlisting}[frame=single, label={subrangetype}, language=pie]
<subrangetype> ::= id     |
                   int    |
                   char
\end{lstlisting}

\begin{lstlisting}[frame=single, label={settype}, language=pie]
<settype> ::= set of <types>
\end{lstlisting}

\begin{lstlisting}[frame=single, label={enumtype}, language=pie]
<enumtype> ::= `(' <idlist> `)'
\end{lstlisting}

\begin{lstlisting}[frame=single, label={recordtype}, language=pie]
<recordtype> ::= record <varlistlist> end
\end{lstlisting}

\begin{lstlisting}[frame=single, label={usertypes}]
<usertypes> ::= `'                     |
                type <listusertypes>
\end{lstlisting}

\begin{lstlisting}[frame=single, label={listusertypes}]
<listusertypes> ::= <usertype> <listusertypesprime>
\end{lstlisting}

\begin{lstlisting}[frame=single, label={listusertypesprime}]
<listusertypesprime> ::= `'               |
                         <listusertypes>
\end{lstlisting}

\begin{lstlisting}[frame=single, label={usertypes}]
<usertype> ::= id `=` <types> `;'
\end{lstlisting}

\begin{lstlisting}[frame=single, label={vars}, language=pie]
<vars> ::= `'                |
           var <varlistlist>
\end{lstlisting}

\begin{lstlisting}[frame=single, label={varlistlist}, language=pie]
<varlistlist> ::= <varlist> <varlistlistprime>
\end{lstlisting}

\begin{lstlisting}[frame=single, label={varlistlistprime}, language=pie]
<varlistlistprime> ::= `'            |
                      <varlistlist>
\end{lstlisting}

\begin{lstlisting}[frame=single, label={varlist}, language=pie]
<varlist> ::= <types> <idlist> `;'
\end{lstlisting}

\begin{lstlisting}[frame=single, label={idlist}, language=pie]
<idlist> ::= id <idattr> <idlistprime>
\end{lstlisting}

\begin{lstlisting}[frame=single, label={idlistprime}, language=pie]
<idlistprime> ::= `'          |
                  `,' <idlist>
\end{lstlisting}

\begin{lstlisting}[frame=single, label={idattr}, language=pie]
<idattr> ::= `'         |
             `=' <expr>
\end{lstlisting}

\begin{lstlisting}[frame=single, label={variable}, language=pie]
<variable> ::= `->' id  <variableprime>                 |
               `[' <exprlistplus> `]' < variableprime>
\end{lstlisting}

\begin{lstlisting}[frame=single, label={variableprime}, language=pie]
<variableprime> ::= `'         |
                    <variable>
\end{lstlisting}

\begin{lstlisting}[frame=single, label={block}, language=pie]
<block> ::= begin <stmts> end
\end{lstlisting}

\begin{lstlisting}[frame=single, label={stmts}, language=pie]
<stmts> ::=  <stmt> <stmtlistprime>
\end{lstlisting}

\begin{lstlisting}[frame=single, label={stmtlistprime}, language=pie]
<stmtlistprime> ::= `' |
                    `;' <stmts>
\end{lstlisting}

\begin{lstlisting}[frame=single, label={stmt}, language=pie]
<stmt> ::= `'             |
           label <stmt>   |
           <block>        |
           <writestmt>    |
           <writelnstmt>  |
           <readstmt>     |
           <readlnstmt>   |
           <loopblock>    |
           <ifblock>      |
           <forblock>     |
           <caseblock>    |
           <gotostmt>     |
           <exitstmt>     |
           <returnstmt>   |
           id <stmtprime>
\end{lstlisting}

\begin{lstlisting}[frame=single, label={stmtprime}, language=pie]
<stmtprime> ::= <attrstmt> |
                <subprogcall>
\end{lstlisting}

\begin{lstlisting}[frame=single, label={subprogcall}, language=pie]
<subprogcall> ::= `(' <exprlist> `)'
\end{lstlisting}

\begin{lstlisting}[frame=single, label={exitstmt}, language=pie]
<exitstmt> ::= exitwhen <expr>
\end{lstlisting}

\begin{lstlisting}[frame=single, label={returnstmt}, language=pie]
<returnstmt> ::= return <expr>
\end{lstlisting}

\begin{lstlisting}[frame=single, label={attrstmt}, language=pie]
<attrstmt> ::= <variable> `:=' <expr> |
               `:=' <expr>
\end{lstlisting}

\begin{lstlisting}[frame=single, label={ifblock}, language=pie]
<ifblock> ::= if <expr> <stmt> <elseblock>
\end{lstlisting}


\begin{lstlisting}[frame=single, label={elseblock}, language=pie]
<elseblock> ::= `'          |
                else <stmt>
\end{lstlisting}

\begin{lstlisting}[frame=single, label={loopblock}, language=pie]
<loopblock> ::= loop <stmt>
\end{lstlisting}

\begin{lstlisting}[frame=single, label={caseblock}, language=pie]
<caseblock> ::= case <expr> of <caselist> <caseblockprime>
\end{lstlisting}

\begin{lstlisting}[frame=single, label={caseblockprime}, language=pie]
<caseblockprime> ::= end |
                     else <stmt> end
\end{lstlisting}

\begin{lstlisting}[frame=single, label={caselist}, language=pie]
<caselist> ::= <literallist> `:' <stmt> `;'
\end{lstlisting}

\begin{lstlisting}[frame=single, label={literallist}, language=pie]
<literallist> ::= <literal> <literallistprime>
\end{lstlisting}

\begin{lstlisting}[frame=single, label={literallistprime}, language=pie]
<literallistprime> ::= `' |
                       `,' <literallist>
\end{lstlisting}

\begin{lstlisting}[frame=single, label={gotostmt}, language=pie]
<gotostmt> ::= goto label
\end{lstlisting}

\begin{lstlisting}[frame=single, label={forblock}, language=pie]
<forblock> ::= for id <forblockprime>
\end{lstlisting}

\begin{lstlisting}[frame=single, label={forblockprime}, language=pie]
<forblockprime> ::= <variable> `:=' <expr> to <expr> step <expr> do <stmt> |
                   `:=' <expr> to <expr> step <expr> do <stmt>
\end{lstlisting}

\begin{lstlisting}[frame=single, label={expr}, language=pie]
<expr> ::= <conj>  <disj>
\end{lstlisting}

\begin{lstlisting}[frame=single, label={disj}, language=pie]
<disj> ::= `' |
           `||' <conj>
\end{lstlisting}

\begin{lstlisting}[frame=single, label={conj}, language=pie]
<conj> ::= <comp> <conjprime>
\end{lstlisting}

\begin{lstlisting}[frame=single, label={conjprime}, language=pie]
<conjprime> ::=  `'           |
                 `&&' <comp>
\end{lstlisting}

\begin{lstlisting}[frame=single, label={comp}, language=pie]
<comp> ::= <relational> <comprime>
\end{lstlisting}

\begin{lstlisting}[frame=single, label={relational}, language=pie]
<relational> ::= <sum> <relationalprime>
\end{lstlisting}

\begin{lstlisting}[frame=single, label={sum}, language=pie]
<sum> ::= <neg> <sumprime>
\end{lstlisting}

\begin{lstlisting}[frame=single, label={sumprime}, language=pie]
<sumprime> ::= `'                       |
              <add_op> <neg> <sumprime>
\end{lstlisting}

\begin{lstlisting}[frame=single, label={neg}, language=pie]
<neg> ::= <mul>         |
          `!' <mul>
\end{lstlisting}

\begin{lstlisting}[frame=single, label={mul}, language=pie]
<mul> ::= <final_term> <mulprime>
\end{lstlisting}

\begin{lstlisting}[frame=single, label={mulprime}, language=pie]
<mulprime> ::= `'                              |
                <mul_op> <final_term> <mulprime>
\end{lstlisting}

\begin{lstlisting}[frame=single, label={relationalprime}, language=pie]
<relationalprime> ::= `'                    |
                      <relational_op> <sum>
\end{lstlisting}

\begin{lstlisting}[frame=single, label={comprime}, language=pie]
<comprime> ::= `' |
              <equality_op> <relational>
\end{lstlisting}

\begin{lstlisting}[frame=single, label={final_term}, language=pie]
<final_term> ::= id <final_termprime> |
                <literal>             |
                `(' <expr> `)'
\end{lstlisting}

\begin{lstlisting}[frame=single, label={final_termprime}, language=pie]
<final_termprime> ::= `'         |
                    <variable>   |
                    <subprogcall>
\end{lstlisting}

\begin{lstlisting}[frame=single, label={add_op}, language=pie]
<add_op> ::= `+' |
             `-'
\end{lstlisting}

\begin{lstlisting}[frame=single, label={mul_op}, language=pie]
<mul_op> ::= `*' |
             `/' |
             `%'
\end{lstlisting}

\begin{lstlisting}[frame=single, label={equality_op}, language=pie]
<equality_op> ::= `==' |
                  `!='
\end{lstlisting}

\begin{lstlisting}[frame=single, label={relational_op}, language=pie]
<relational_op> ::= `<' |
	   	    `<=' |
		    `>'  |
		    `>='
\end{lstlisting}

\begin{lstlisting}[frame=single, label={literal}, language=pie]
<literal> ::= intliteral        |
              realiteral        |
              charliteral       |
              stringliteral     |
              <subrangeliteral>
\end{lstlisting}

\begin{lstlisting}[frame=single, label={exprlist}, language=pie]
<exprlist> ::= `' |
               <exprlistplus>
\end{lstlisting}

\begin{lstlisting}[frame=single, label={exprlistplus}, language=pie]
<exprlistplus> ::= <expr> <exprlistplusprime>
\end{lstlisting}

\begin{lstlisting}[frame=single, label={exprlistplusprime}, language=pie]
<exprlistplusprime> ::= `'              |
                     `,' < exprlistplus>
\end{lstlisting}

\begin{lstlisting}[frame=single, label={subprograms}, language=pie]
<subprograms> ::= `'                               |
                  <procedure> <subprogramsprime>   |
                  <function>  <subprogramsprime>
\end{lstlisting}

\begin{lstlisting}[frame=single, label={subprogramsprime}, language=pie]
<subprogramsprime> ::= `'                |
                       `;' <subprograms>
\end{lstlisting}

\begin{lstlisting}[frame=single, label={procedure}, language=pie]
<procedure> ::= proc id `(' <param> `)' `;' <decl> <block>
\end{lstlisting}

\begin{lstlisting}[frame=single, label={function}, language=pie]
<function> ::= func <types> id `(' <param> `)' `;' <decl> <block>
\end{lstlisting}

\begin{lstlisting}[frame=single, label={param}, language=pie]
<param> ::= `'            |
            <paramlistlist>
\end{lstlisting}

\begin{lstlisting}[frame=single, label={paramlistlist}, language=pie]
<paramlistlist> ::= `' |
                    `;' <paramlistlist>
\end{lstlisting}

\begin{lstlisting}[frame=single, label={paramlistlist}, language=pie]
<paramlist> ::= <types> <idlist> | 
                ref <types> <idlist>
\end{lstlisting}

\begin{lstlisting}[frame=single, label={paramlistlistprime}, language=pie]
<paramlistlistprime> ::= <paramlist> <paramlistlistprime>
\end{lstlisting}

\begin{lstlisting}[frame=single, label={writestmt}, language=pie]
<writestmt> ::= write `(' <expr> `)'
\end{lstlisting}

\begin{lstlisting}[frame=single, label={writelnstmt}, language=pie]
<writelnstmt> ::= writeln `(' <expr> `)'
\end{lstlisting}

\begin{lstlisting}[frame=single, label={readstmt}, language=pie]
<readstmt> ::= read `(' id <variableprime> `)'
\end{lstlisting}

\begin{lstlisting}[frame=single, label={readlnstmt}, language=pie]
<readlnstmt> ::= readln `(' id <variableprime> `)'
\end{lstlisting}
\end{footnotesize}


\newpage
\chapter{Termos léxicos}
\begin{verbatim}
    ; ( ) [ ] { } nil program proc begin end func const type var if else goto of
    for to do step in loop exitwhen case write writeln read readln return
    int bool real char string array set record enum subrange
\end{verbatim}
\section{Operadores aritméticos (numpericop)}
\begin{verbatim}
    +  -  *  /  % 
\end{verbatim}

\section{Operadores de conjuntos}
\begin{Verbatim}[commandchars=\\\{\}]
    + - * = != <= in
\end{Verbatim}

\section{Operadores de declaração}
\begin{verbatim}
    =
\end{verbatim}

\section{Operadores de atribuição}
\begin{Verbatim}[commandchars=\\\{\}]
    :=
\end{Verbatim}

\section{Operadores de comparação (boolop)}
\begin{Verbatim}[commandchars=\\\{\}]
    >  <  >=  <=  == !=
\end{Verbatim}

\section{Operadores lógicos (boolop)}
\begin{Verbatim}[commandchars=\\\{\}]
    && || !
\end{Verbatim}

\section{Literais booleanos}
\begin{verbatim}
    true false
\end{verbatim}

\newpage
\chapter{Regras sobre símbolos terminais}
Meta-operadores para definir as expressões regulares:
\begin{verbatim}
    [ ] : para enumerações associadas à -
    *   : repetição
    +   : para repetições de um vez ou mais
    ?   : de zero a uma vez
    .   : como um caractere ``joker'' exceto \n
    ^   : para o complementar de um  [ ]
    |   : para representar uma alternativa
\end{verbatim}

\section{id}
Identificadores podem começar apenas com letras, podem ter números e underline (``\_'') em sua estrutura.
A expressão regular que gera um identificador correto é:
 \begin{verbatim}
   id :  [a-zA-Z][a-zA-Z0-9_]*
\end{verbatim}

\section{label}
 \begin{verbatim}
   label : "@"[a-zA-Z0-9_]*
\end{verbatim}

\section{char}
\begin{verbatim}
   charliteral : \`[^']*\'
\end{verbatim}

\section{string}
\begin{verbatim}
   stringliteral : \"[^"\n]*\"
\end{verbatim}

\section{int}
\begin{verbatim}
   intliteral :  (("-"|"+")?[0-9]+)
\end{verbatim}

\section{real}
\begin{verbatim}
   exponent : ([E|e]("+"|"-")?({DIGIT}+))
   real : ([0-9]*[.])?[0-9]+
   realexponent : ([0-9]*[.])?[0-9]+{exponent}?
   realliteral :  (("-"|"+")?{real}|("-"|"+")?{realexponent})
\end{verbatim}

\section{Precedência}
A ordem de precedência deve valer para os seguintes operadores (), [], \{\}, $*$, $/$, \%, !, +, -, $<$, $<=$, $>$, $>=$, $:=$, $=$, $==$, $!=$, $\&\&$, $||$. A ordem de precedência pode ser visualizada na Tabela \ref{table:ordem_precedencia}.

\begin{table*}[h]
\renewcommand{\arraystretch}{1.34}
\centering
\begin{tabular}{| c | c | c | c | c | c |}
\hline
\bfseries Operador & \bfseries Precedência  \\
\hline
() & maior \\ \hline
[] &  \\ \hline
\{\} &  \\ \hline
 *, /, \% & \\ \hline
  ! & \\ \hline
 +, - & \\ \hline
$<$, $<=$, $>$, $>=$ & \\ \hline
$==$, $!=$ & \\ \hline
\&\& & \\ \hline
$||$ & \\ \hline
$:=$, = & menor\\ \hline
\end{tabular}
\caption{Ordem de precedência para os operadores.}
\label{table:ordem_precedencia}
\end{table*}

% Pascal tem funções matemáticas na própria linguagem (vamos abortar ou trataremos como se fosse de uma biblioteca?)
\section{Regra para comentários}
    Comentários iniciam por ``\#'' e são eliminados no pré-processamento.
\begin{verbatim}
   linecomment : "#"((.)*)\n
\end{verbatim}


\newpage
\chapter{Analisadores Sintáticos Descendentes}
Os dois parsers implementados (recursivo e por tabela) são parsers recursivos. Isso significa que não utilizam de backtracking e portanto rodam em tempo linear (de acordo com o tamnho do input, a palavra sendo analisada). Isso é importante pois o tempo de compilação de programas deve ser o menor possível. O parsing é feito analisando as regras sintáticas da linguagem e um input (uma palavra), e é utilizado para determinar se essa palavra faz parte da linguagem. 

\section{Conjuntos Predict}
Cada regra sintática tem associada a si um conjunto de terminais chamado \textit{Predict}.

Seja $A$ um símbolo não-terminal, $S$ o símbolo não-terminal inicial, $\alpha$ e $\beta$ sequências de não-terminais e/ou terminais e $T$ o conjunto dos símbolos terminais,

\begin{equation*}
Predict(A\rightarrow \alpha) = 
\begin{cases}
First(\alpha) \cup Follow(A)  & \text{ if } A \Rightarrow^* \lambda  \\
First(\alpha) & \text{otherwise}
\end{cases}
\end{equation*}

\begin{equation*}
First( \alpha) = \{s \in T \mid \alpha \Rightarrow^* s\beta\}
\end{equation*}

\begin{equation*}
Follow(A) = \{s \in T \mid S \Rightarrow^* ...As...\}
\end{equation*}

O conjunto Predict de uma regra $\alpha$ basicamente nos diz quais símbolos terminais são candidatos a serem o próximo a aparecer se usarmos $\alpha$. Tanto no parser recursivo como o por tabela, a cada passo nós estamos analisando um símbolo não-terminal $A$ (começando pelo símbolo inicial) e um terminal $s$ (vindo do input). Para sabermos qual regra utilizar, basta olharmos as regras de $A$ e acharmos a que tem $s$ no seu conjunto Predict. Se $s$ não estiver no conjunto Predict de nenhuma regra de $A$, o input não é uma palavra da linguagem, e um erro deve ser gerado. Como nossa gramática é LL1 (ignorando o caso do else), isso significa que não há interseção nos conjuntos Predict de regras de um mesmo símbolo não-terminal. Ou seja, para cada não-terminal $A$ e terminal $s$, no máximo uma regra de $A$ terá $s$ no seu Predict, permitindo que o parsing seja realizado da maneira descrita nesse parágrafo com sucesso. Os conjuntos Predict de todas as regras da nossa gramática podem ser vistos na Tabela \ref{tab:predict}.

\begin{center}
\begin{longtable}[H]{|C{5cm} | C{5cm} | C{5cm} |}
\caption{Predict}\label{tab:predict}\\
\hline
\textbf{Nonterminal} & \textbf{Rule} & \textbf{Predict} \\
\hline
PROGRAM &`program' `id' `;' DECL BLOCK `.' & \{`program'\} \\
\hline
DECL &CONSTS USERTYPES VARS SUBPROGRAMS &\{`const', `type', `var', `begin', `func', `proc'\} \\
\hline
\multirow{2}{*}{CONSTS} & `' & \{`var', `proc', `func', `type', `begin'\} \\ \cline{2-3}
&`const' LISTCONST & \{`const'\} \\
\hline
LISTCONST & CONSTDECL LISTCONSTPRIME & \{`id'\}\\
\hline
\multirow{2}{*}{LISTCONSTPRIME} & `' & \{`var', `proc', `func', `type', `begin'\} \\ \cline{2-3}
& LISTCONST & \{`id'\}\\
\hline
CONSTDECL & `id' `=' EXPR `;' & \{`id'\}\\
\hline
\multirow{2}{*}{TYPES} & `id' TYPESPRIME & \{`id'\} \\ \cline{2-3}
& PRIMTYPES & \{`int', `real', `bool', `char', `string', `array', `set', `(', `record'\}\\
\hline
\multirow{2}{*}{TYPESPRIME} & `' & \{`id', `;', \} \\ \cline{2-3}
& `..' SUBRANGETYPE & \{`..'\}\\
\hline
\multirow{9}{*}{PRIMTYPES} & `int' TYPESPRIME & \{`int' \} \\ \cline{2-3}
& `real' & \{`real'\}  \\ \cline{2-3}
& `bool' & \{`bool'\} \\ \cline{2-3}
& `char' & \{`char'\} \\ \cline{2-3}
& `string' & \{`string'\} \\ \cline{2-3}
& ARRAYTYPE & \{`array'\} \\ \cline{2-3}
& SETTYPE & \{`set'\} \\ \cline{2-3}
& ENUMTYPE & \{`('\} \\ \cline{2-3}
& RECORDTYPE & \{`record'\} \\ \cline{2-3}
\hline
ARRAYTYPE & `array' `[' SUBRANGELIST `]' `of' TYPES & \{`array'\} \\
\hline
\multirow{3}{*}{SUBRANGELIST} & `id' SUBRANGEPRIME & \{`id'\} \\ \cline{2-3}
& `int' '..' SUBRANGETYPE SUBRANGELISTPRIME & \{`int'\} \\ \cline{2-3}
& `char' '..' SUBRANGETYPE SUBRANGELISTPRIME & \{`char'\} \\
\hline
\multirow{2}{*}{SUBRANGEPRIME} & SUBRANGELISTPRIME & \{`]', `,'\} \\ \cline{2-3}
& `..' SUBRANGETYPE SUBRANGELISTPRIME & \{`..'\}\\
\hline
\multirow{2}{*}{SUBRANGELISTPRIME} & `' & \{`]'\} \\ \cline{2-3}
& `,' SUBRANGELIST & \{`,'\}\\
\hline
\multirow{3}{*}{SUBRANGETYPE} & `id' & \{`id'\} \\ \cline{2-3}
& `int' & \{`int'\} \\ \cline{2-3}
& `char' & \{`char'\}\\
\hline
SETTYPE & `set' `of' TYPES & \{`set'\}\\
\hline
ENUMTYPE & `(' IDLIST `)' & \{`('\} \\
\hline
RECORDTYPE & `record' VARLISTLIST `end' & \{`record'\} \\
\hline
\multirow{2}{*}{USERTYPES} & `' & \{`var', `begin', `proc', `func'\} \\ \cline{2-3}
& `type' LISTUSERTYPES & \{`type'\}\\
\hline
LISTUSERTYPES & USERTYPE LISTUSERTYPESPRIME & \{`id'\} \\
\hline
\multirow{2}{*}{LISTUSERTYPESPRIME} & `' & \{`var', `begin', `proc', `func'\} \\ \cline{2-3}
& LISTUSERTYPES & \{`id'\}\\
\hline
USERTYPE & `id' `=' TYPES `;' & \{`id'\}\\
\hline
\multirow{2}{*}{VARS} & `' & \{`begin', `proc', `func'\} \\ \cline{2-3}
& `var' VARLISTLIST & \{`var'\}\\
\hline
VARLISTLIST & VARLIST VARLISTLISTPRIME & \{`id', `int', `real', `bool', `char', `string', `array', `set', `(', `record'\}\\
\hline
\multirow{2}{*}{VARLISTLISTPRIME} & `' & \{`end', `begin', `proc', `func'\} \\ \cline{2-3}
& VARLISTLIST & \{`id', `int', `real', `bool', `char', `string', `array', `set', `(', `record'\}\\
\hline
VARLIST & TYPES IDLIST `;' & \{`id', `int', `real', `bool', `char', `string', `array', `set', `(', `record'\} \\
\hline
IDLIST & `id' IDATTR IDLISTPRIME & \{`id'\} \\
\hline
\multirow{2}{*}{IDLISTPRIME} & `' & \{`;', `)'\} \\ \cline{2-3}
& `,' IDLIST & \{`,'\}\\
\hline
\multirow{2}{*}{IDATTR} & `' & \{`;', `,', `)'\} \\ \cline{2-3}
& `=' EXPR & \{`='\}\\
\hline
\multirow{2}{*}{VARIABLE} & `$=>$' `id' VARIABLEPRIME & \{`$=>$'\} \\ \cline{2-3}
& `[' EXPRLISTPLUS `]' VARIABLEPRIME & \{`['\}\\
\hline
\multirow{2}{*}{VARIABLEPRIME} & `' `id' VARIABLEPRIME & \{`id', `;', `]', `of', `,', `)', `end', `begin', `label', `exitwhen', `return', `:=', `if', `else', `loop', `case', `goto', `for', `to', `step', `do', `$||$', `\&\&', `+', `-', `*', `/', `\%', `$==$', `$!=$', `<', `$<=$', `<', `$>=$', `write', `writeln', `read', `readln'\} \\ \cline{2-3}
& VARIABLE & \{`[', `$=>$'\}\\
\hline
BLOCK & `begin' STMTS `end' & \{`begin'\} \\
\hline
STMTS & STMT STMTLISTPRIME & \{`id', `;', `end', `begin', `label', `exitwhen', `return', `if', `loop', `case', `goto', `for', `write', `writeln', `read', `readln'\} \\
\hline
\multirow{2}{*}{STMTLISTPRIME} & `' & \{`end'\} \\ \cline{2-3}
& `;' STMTS & \{`;'\}\\
\hline
\multirow{15}{*}{STMT} & `' & \{`;', `end', `else'\} \\ \cline{2-3}
& `label' STMT & \{`label'\}\\ \cline{2-3}
& `BLOCK & \{`begin'\}\\ \cline{2-3}
& WRITESTMT & \{`write'\}\\ \cline{2-3}
& WRITELNSTMT & \{`writeln'\}\\ \cline{2-3}
& READSTMT & \{`read'\}\\ \cline{2-3}
& READLNSTMT & \{`readln'\}\\ \cline{2-3}
& LOOPBLOCK & \{`loop'\}\\ \cline{2-3}
& IFBLOCK & \{`if'\}\\ \cline{2-3}
& FORBLOCK & \{`for'\}\\ \cline{2-3}
& CASEBLOCK & \{`case'\}\\ \cline{2-3}
& GOTOSTMT & \{`goto'\}\\ \cline{2-3}
& 'id' STMTPRIME & \{`id'\}\\ \cline{2-3}
& EXITSTMT & \{`exitwhen'\}\\ \cline{2-3}
& RETURNSTMT & \{`return'\}\\
\hline
\multirow{2}{*}{STMTPRIME} & ATTRSTMT & \{`[', `:=', `-$>$ '\} \\ \cline{2-3}
& SUBPROGCALL & \{`('\} \\
\hline
SUBPROGCALL & `(' EXPRLIST `)' & \{`('\} \\
\hline
EXITSTMT & `exitwhen' EXPR & `exitwhen'\\
\hline
RETURNSTMT & `return' EXPR & `return'\\
\hline
\multirow{2}{*}{ATTRSTMT} & VARIABLE ':=' EXPR & \{`[', `-$>$'\}\\ \cline{2-3}
& ':=' EXPR & \{`:='\}\\
\hline
IFBLOCK & 'if' EXPR STMT ELSEBLOCK & \{`if'\}\\
\hline
\multirow{2}{*}{ELSEBLOCK} & `' & \{`;', `end'\}\\ \cline{2-3}
& `else' STMT & \{`else'\}\\
\hline
LOOPBLOCK & `loop' STMT & \{`loop'\}\\
\hline
CASEBLOCK & `case' EXPR 'of' CASELIST CASEBLOCKPRIME & \{`case'\}\\
\hline
\multirow{2}{*}{CASEBLOCKPRIME} & `end' & \{`end'\}\\ \cline{2-3}
& `else' STMT `end' & \{`else'\}\\
\hline
CASELIST & LITERALLIST `:' STMT `;' & \{`intliteral', `realliteral', `charliteral', `stringliteral', `subrangeliteral'\}\\
\hline
LITERALLIST & LITERAL LITERALLISTPRIME & \{`intliteral', `realliteral', `charliteral', `stringliteral', `subrangeliteral'\}  \\
\hline
\multirow{2}{*}{LITERALLISTPRIME} & `' & \{`:'\}  \\ \cline{2-3}
& `,' LITERALLIST & \{`,'\}\\
\hline
GOTOSTMT & `goto' `label' & \{`goto'\}\\
\hline
FORBLOCK & `for' `id' FORBLOCKPRIME & \{`for'\}\\
\hline
\multirow{2}{*}{FORBLOCKPRIME} & VARIABLE `:=' EXPR `to' EXPR `step' EXPR `do' STMT & \{`[', `-$>$'\}\\ \cline{2-3}
& `:=' EXPR `to' EXPR `step' EXPR `do' STMT & \{`:='\}\\
\hline
EXPR & CONJ DISJ & \{`id', `(', `!', `intliteral', `realliteral', `charliteral', `stringliteral', `subrangeliteral'\}\\
\hline
\multirow{3}{*}{FINAL\_TERM} & `id' FINAL\_TERMPRIME & \{`id'\}\\ \cline{2-3}
& LITERAL & \{`intliteral', `realliteral', `charliteral', `stringliteral', `subrangeliteral'\}\\ \cline{2-3}
& `(' EXPR `)' & \{`('\}\\
\hline
\multirow{3}{*}{FINAL\_TERMPRIME} & VARIABLE & \{`[', `-$>$'\}\\ \cline{2-3}
& `' & \{`id', `;', `]', `of', `,', `)', `end', `begin', `label', `exitwhen', `return', `if', `else', `loop', `case', `goto', `for', `to', `step', `do', `$\mid \mid$', `\&\&', `+', `-', `*', `/', `\%', `==', `!=', `$<$', `$<$=', `$>$', `=$>$', `write', `writeln', `read', `readln'\}\\ \cline{2-3}
& SUBPROGCALL & \{`('\}\\
\hline
\multirow{2}{*}{DISJ} & `' & \{`id', `;', `]', `of', `,', `)', `end', `begin', `label', `exitwhen', `return', `if', `else', `loop', `case', `goto', `for', `to', `step', `do', `write', `writeln', `read', `readln'\}\\ \cline{2-3}
& `$\mid \mid$' CONJ & \{`$\mid \mid$'\}\\
\hline
CONJ & COMP CONJPRIME & \{`id', `(', `!', `intliteral', `realliteral', `charliteral', `stringliteral', `subrangeliteral'\}\\
\hline
\multirow{2}{*}{CONJPRIME} & `' & \{`id', `;', `]', `of', `,', `)', `end', `begin', `label', `exitwhen', `return', `if', `else', `loop', `case', `goto', `for', `to', `step', `do', `$\mid \mid$', `write', `writeln', `read', `readln'\} \\
\cline{2-3}
& `\&\&' COMP & \{`\&\&'\} \\
\hline
COMP & RELATIONAL COMPPRIME & \{`id', `(', `!', `intliteral', `realliteral', `charliteral', `stringliteral', `subrangeliteral'\} \\
\hline
RELATIONAL & SUM RELATIONALPRIME & \{`id', `(', `!', `intliteral', `realliteral', `charliteral', `stringliteral', `subrangeliteral'\} \\
\hline
\multirow{2}{*}{RELATIONALPRIME} & `' & \{`id', `;', `]', `of', `,', `)', `end', `begin', `label', `exitwhen', `return', `if', `else', `loop', `case', `goto', `for', `to', `step', `do', `$\mid \mid$', `==', `!=', `\&\&', `write', `writeln', `read', `readln'\} \\
\cline{2-3}
& RELATIONAL\_OP SUM & \{`$<$', `$<$=', `$>$', `=$>$'\} \\
\hline
\multirow{2}{*}{COMPPRIME} & `' & \{`id', `;', `]', `of', `,', `)', `end', `begin', `label', `exitwhen', `return', `if', `else', `loop', `case', `goto', `for', `to', `step', `do', `$\mid \mid$', `\&\&', `write', `writeln', `read', `readln'\} \\
\cline{2-3}
& EQUALITY\_OP RELATIONAL & \{`==', `!='\} \\
\hline
SUM & NEG SUMPRIME & \{`id', `(', `!', `intliteral', `realliteral', `charliteral', `stringliteral', `subrangeliteral'\} \\
\hline
\multirow{2}{*}{SUMPRIME} & '' & \{`id', `;', `]', `of', `,', `)', `end', `begin', `label', `exitwhen', `return', `if', `else', `loop', `case', `goto', `for', `to', `step', `do', `$\mid \mid$', `==', `!=' `\&\&', `$<$', `$<$=', `$>$', `=$>$', `write', `writeln', `read', `readln'\} \\
\cline{2-3}
& ADD\_OP NEG SUMPRIME & \{`+', `-'\} \\
\hline
\multirow{2}{*}{NEG} & MUL & \{`id', `(', `intliteral', `realliteral', `charliteral', `stringliteral', `subrangeliteral'\} \\
\cline{2-3}
& '!' MUL & \{`!'\} \\
\hline
MUL & FINAL\_TERM MULPRIME & \{`id', `(', `intliteral', `realliteral', `charliteral', `stringliteral', `subrangeliteral'\}\\
\hline
\multirow{2}{*}{MULPRIME} & `' & \{`id', `;', `]', `of', `,', `)', `end', `begin', `label', `exitwhen', `return', `if', `else', `loop', `case', `goto', `for', `to', `step', `do', `$||$', `$+$', `$-$', `==', `!=', `$<$', `$<=$', `$>$', `$>=$', `\&\&', `write', `writeln', `read', `readln'\} \\
\cline{2-3}
& MUL\_OP FINAL\_TERM MULPRIME & \{`$*$', `$/$', `$\%$'\} \\
\hline
\multirow{2}{*}{ADD\_OP} & `$+$' & \{`$+$'\} \\
\cline{2-3}
& `$-$' & \{`$-$'\} \\
\cline{2-3}
\hline
\multirow{3}{*}{MUL\_OP} & `$*$' & \{`$*$'\} \\
\cline{2-3}
& `$/$' & \{`$/$'\} \\
\cline{2-3}
& `$\%$' & \{`$\%$'\} \\
\cline{2-3}
\hline
\multirow{2}{*}{EQUALITY\_OP} & `$==$' & \{`$==$'\} \\
\cline{2-3}
& `$!=$' & \{`$!=$'\} \\
\cline{2-3}
\hline
\multirow{4}{*}{RELATIONAL\_OP} & `$<$' & \{`$<$'\} \\
\cline{2-3}
& `$<=$' & \{`$<=$'\} \\
\cline{2-3}
& `$>$' & \{`$>$'\} \\
\cline{2-3}
& `$>=$' & \{`$>=$'\} \\
\hline
\multirow{5}{*}{LITERAL} & `intliteral' & \{`intliteral'\} \\
\cline{2-3}
& `realliteral' & \{`realliteral'\} \\
\cline{2-3}
& `charliteral' & \{`charliteral'\} \\
\cline{2-3}
& `stringliteral' & \{`stringliteral'\} \\
\cline{2-3}
& `subrangeliteral' & \{`subrangeliteral'\} \\
\hline
\multirow{2}{*}{EXPRLIST} & `' & \{`)'\} \\
\cline{2-3}
& EXPRLISTPLUS & \{`id', `(', `!', `intliteral', `realliteral', `charliteral', `stringliteral', `subrangeliteral'\} \\
\hline
EXPRLISTPLUS & EXPR EXPRLISTPLUSPRIME & \{`id', `(', `!', `intliteral', `realliteral', `charliteral', `stringliteral', `subrangeliteral'\} \\
\hline
\multirow{2}{*}{EXPRLISTPLUSPRIME} & `' & \{`]', `)'\} \\
\cline{2-3}
&  `,' EXPRLISTPLUS & \{`,'\} \\
\hline
\multirow{3}{*}{SUBPROGRAMS} & `' & \{`begin'\} \\
\cline{2-3}
& PROCEDURE SUBPROGRAMSPRIME & \{`proc'\} \\
\cline{2-3}
& FUNCTION SUBPROGRAMSPRIME & \{`func'\} \\
\hline
\multirow{2}{*}{SUBPROGRAMSPRIME} & `' & \{`begin'\} \\
\cline{2-3}
& ';' SUBPROGRAMS & \{`;'\} \\
\hline
PROCEDURE & 'proc' 'id' '(' PARAM ')' ';' DECL BLOCK & \{`proc'\} \\
\hline
FUNCTION & 'func' TYPES 'id' '(' PARAM ')' ';' DECL BLOCK & \{`func'\} \\
\hline
\multirow{2}{*}{PARAM} & `' & \{`)'\} \\
\cline{2-3}
& PARAMLISTLIST & \{`id', `int', `real', `bool', `char', `string', `array', `set', `(', `record'\} \\
\hline
PARAMLISTLIST & PARAMLIST PARAMLISTLISTPRIME & \{`id', `int', `real', `bool', `char', `string', `array', `set', `(', `record'\} \\
\hline
\multirow{2}{*}{PARAMLISTLISTPRIME} & `' & \{`)'\} \\
\cline{2-3}
& `;' PARAMLISTLIST & \{`;'\} \\
\hline
\multirow{2}{*}{PARAMLIST} & `ref' TYPES IDLIST & \{`)'\} \\
\cline{2-3}
& TYPES IDLIST & \{`id', `int', `real', `bool', `char', `string', `array', `set', `(', `record'\} \\
\hline
WRITESTMT & `write' `(' EXPR `)' & \{`write'\} \\
\hline
WRITELNSTMT & `writeln' `(' EXPR `)' & \{`writeln'\} \\
\hline
READSTMT & `read' `(' `id' VARIABLEPRIME `)' &  \{`read'\} \\
\hline
READLNSTMT & `readln' `(' `id' VARIABLEPRIME `)' & \{`readln'\} \\
\hline
\end{longtable}
\end{center}

\section{Parser preditivo recursivo}
No parser recursivo, é utilizada a pilha de execução do programa para fazer o parsing top-down. Há uma funcão para cada não-terminal e o programa é inicilizado chamando a função do não-terminal inicial. Na função de cada não-terminal $A$, fazemos um switch/case utilizando o próximo símbolo terminal $t$ainda não \textit{matched} do input, onde cada case é a regra tal que $t$ está no Predict dessa regra: para cada não-terminal $B$ da regra, chamamos a função de $B$, e para cada terminal $s$ da regra, chamamos um \textit{matching} de $s$ e $t$ (que gera um erro se $s$ e $t$ forem diferentes e avança no input para o próximo símbolo). O caso default é onde $t$ não está no Predict de nenhuma regra de $A$, e gera um erro.  

\section{Parser preditivo por tabela}
No parser por tabela, utilizamos uma pilha explícita e uma tabela. No nosso código, a tabela é um std::map onde a chave é um par de um símbolo não-terminal e um símbolo terminal e o valor é um vetor de símbolos terminais e não-terminais (uma regra). Essa tabela é construída utilizando os conjuntos Predict; há uma entrada na tabela para a chave ($A$,$s$) se $s$ está no Predict de alguma regra de $A$, e o valor associado a essa chave é justamente essa regra. O programa é inicilizado empilhando o símbolo não-terminal inicial, e então fica em um loop enquanto a pilha não estiver vazia. A cada passo desse loop, olhamos o símbolo no topo da pilha. Se for um símbolo terminal $s$, fazemos o \textit{matching} com o próximo símbolo terminal $t$ ainda não \textit{matched} do input (que gera um erro se $s$ e $t$ forem diferentes e avança no input para o próximo símbolo). Se for um símbolo não-terminal $A$, olhamos na tabela a chave {$A$,$t$} e empilhamos a regra (o vetor) correspondente. Caso não haja um vetor para essa chave, um erro é gerado.

\section{Recuperação de erro}
O compilador não deve parar sua execução ao encontrar um erro sintático, mas sim notificar o erro e continuar o parsing. Desse modo, vários erros podem ser reportados com uma só compilação. Para isso, ao encontrar um erro, é preciso utilizar de alguma técnica de recuperação de erro para que o parsing possa continuar. No nosso caso, utilizamos diferentes técnicas de recuperação de erro. 

No parser recursivo, quando há um erro de \textit{matching} ou um erro de símbolo não esperado (quando não há regra cujo Predict contém o símbolo), esse erro é notificado e então avançamos no input para o próximo símbolo. Isso é uma boa técnica quando erros de digitação são frequentes, mas resulta em muitos outros erros se há uma palavra faltando no input. 

Já no parser por tabela, o erro de símbolo não esperado é recuperado de uma forma diferente: desempilhamos a pilha até encontrarmos um $;$ (ou até ela ficar vazia) e avançamos no input até encontrarmos um $;$ (ou até ele acabar). A intuição é que estamos continuando o parsing a partir do próximo $;$. O erro de \textit{matching} é recuperado da mesma forma que no recursivo.

\section{Saídas dos parsers}
Os parsers não tem saída quando não há erros, ou seja, quando o programa (o input) for uma palavra da linguagem. Para todos os nossos programas de exemplo (factorial.pie, quicksort.pie e mergesort.pie), os dois parsers não estão imprimindo nenhuma saída, o que significa que esses programas estão corretos em relação a linguagem.

Quando há um erro, ele é disponibilizado das seguintes formas:
\vspace{0.5cm}

\begin{verbatim}
ERROR: At row 48 column 24 
    Not expected symbol: ;
ERROR: At row 49 column 17 
    Not expected symbol: ID_TOKEN (with lexeme k)
ERROR: At row 49 column 19
    Expected: END_TOKEN
    Found: ATTR_TOKEN
\end{verbatim} 




\newpage
\chapter{Analisador Sintático Ascendente}
Foi utilizada a ferramenta yacc para construção desse parser. O yacc é uma ferramenta que gera um parser LALR(1). Este tipo de parser contém uma pilha e um input, e executa ações de acordo com uma tabela, considerando o estado no topo da pilha e o símbolo lookahead. Essas ações podem ser shift (avançar o input e empilhar um estado), reduce (desempilhar estados de uma regra, olhar tabela com estado agora no topo e símbolo não terminal da regra, empilhar o estado que a tabela informou), aceitar e erro. A construção dessa tabela, que é o que vai ditar qual ação deve ser executada, é feita a partir de um grafo de estados, onde cada estado contém uma lista de itens (regras com uma posição demarcada e um símbolo lookahead). A diferença nesse grafo ou na construção da tabela a partir dos grafos é o que diferencia os diversos parsers LR.

\section{Conflitos}
O yacc avisa quando há conflitos no parsing, ou seja, quando há mais de uma possível ação. Nossa gramática é ambígua na regra do "else", o que causa um conflito shift/reduce (qualquer uma das duas ações poderiam ser executadas). O padrão do yacc é resolver esse conflito com o shift, e no nosso caso é isso que queremos pois desse modo a interpretação é que o "else" é combinado com o "if" mais recente, então não fazemos nada e deixamos o yacc mesmo resolver.

\section{Saída do Analisador}
Na saída padrão, o analisador informa os erros encontrados no programa. Se não houver error, não há nada na saída padrão, que é o caso para os exemplos. Ao informar um erro, o analisador indica qual a sua linha e coluna, além de qual token não era esperado.
Além disso, o output do analisador inclui um arquivo de texto, que contém o pretty printing do programa de entrada; isto é, uma impressão "bonita" do programa de entrada, toda devidamente identada. O pretty printing pode não ocorrer corretamente quando há erros no programa. Para todos os exemplos, o pretty printing é feito perfeitamente.

\section{Recuperação de Erro}
Para recuperação de erro, foi usado o token "error" do próprio yacc, que ele reconhece como um símbolo terminal e desempilha a pilha e descarta símbolos da entrada até que possa retornar ao parsing normal. Esse token é usado criando novas regras e o adicionando onde erros são esperados. No nosso caso, adicionamos esse token antes do ';' que separa statements, por exemplo. Desse modo, quando há um erro em um statement, as coisas entre o erro e o próximo ';' (que simboliza o fim desse statement) são ignoradas. Dessa forma, quando um erro é encontrado ele ainda é reportado, mas a execução do programa não é interrompida, podendo informar sobre diversos erros de uma vez. O único problema é que, por ignorar algumas coisas depois de um erro, também ignora algumas ações semânticas do pretty printing, fazendo com que este não fique certo. Portanto o pretty printing deve ser ignorado até que todos os erros sejam consertados.

\newpage
\chapter{Forma de uso}
O código se encontra em \url{https://github.com/raquel-oliveira/PIE}
\section{Compilação}
\begin{verbatim}
    make
\end{verbatim}

\section{Execução}
\subsection{Parser ascendente}
\begin{verbatim}
    ./pie pathtofile.pie
\end{verbatim}

\section{Exemplo}
\begin{verbatim}
    make
    ./pie codesamples/mergesort.pie
\end{verbatim}

\newpage
\chapter{Exemplos de programas}
%http://rextester.com/GHRH16649
\section{MergeSort}
\begin{footnotesize}
\lstinputlisting{../codesamples/merge_sort.pie}
\end{footnotesize}

%http://sandbox.mc.edu/~bennet/cs404/doc/qsort_pas.html
\section{Quicksort}
\begin{footnotesize}
\lstinputlisting{../codesamples/quick_sort.pie}
\end{footnotesize}

%http://rextester.com/UXP1971
\section{Fatorial}
\begin{footnotesize}
\lstinputlisting{../codesamples/factorial.pie}
\end{footnotesize}

\newpage
\chapter{Participação}
\begin{table}[h]
	\centering
	\label{tab:participation}
	\begin{tabular}{|l|l|r|}
		\hline
		\textbf{Matrícula} & \textbf{Aluno(a)} & \textbf{Participação (\%)} \\ \hline
		20180008183        & BRENO MAURICIO DE FREITAS VIANA & 20\%                       \\ \hline
		20170153995        & FERNANDA MENEZES PAES ISABEL    & 20\%                       \\ \hline
		2012912517         & LUIZ ARTHUR DE LIMA FREIRE      & 20\%                       \\ \hline
		20180008281        & RAQUEL LOPES DE OLIVEIRA        & 20\%                       \\ \hline
		20180008316        & VÍTOR DE GODEIRO MARQUES        & 20\%                       \\ \hline
		\multicolumn{2}{|l|}{\textbf{TOTAL}}            & 100\%                      \\ \hline
	\end{tabular}
\end{table}

\nocite{*}
\bibliographystyle{ieeetr}
\bibliography{bibpie}


\end{document}

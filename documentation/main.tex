\documentclass[12pt]{report}
\usepackage[utf8]{inputenc}
\usepackage[brazilian]{babel}
\usepackage[protrusion=true,expansion=true]{microtype} % S2  %
\usepackage{listings}
\usepackage{fullpage} % changes the margin
\usepackage{pxfonts} % to bold in listings
\usepackage{xcolor}
\usepackage{fancyvrb}
\usepackage{float}
\usepackage{comment}

\def\changemargin#1#2{\list{}{\rightmargin#2\leftmargin#1}\item[]}
\let\endchangemargin=\endlist

%definition of language PIE
\lstdefinelanguage{pie}
{
    morekeywords={[1]
        program,
        proc,
        begin,
        end,
        func,
        const,
        type,
        var
    },
    morekeywords={[2]
        if,
        else,
        goto,
        of, %array, set, subrange
        for,
        to, %for
        do, %for
        step, %for
        in, %subrange, set
        loop,
        exitwhen, %loop
        case,
        write,
        writeln,
        read,
        readln,
        return,
        ref
    },
    morekeywords={[3]
        int,
        bool,
        real,
        char,
        string,
        array,
        set,
        record,
        enum,
        subrange
    },
    morekeywords={[4]
        true, false, nil
    },
    morekeywords={[5]
        id,
        label,
        intliteral,
        realiteral,
        charliteral,
        stringliteral
    },
    sensitive=true,
    morecomment=[l]{\#},
    morestring=[b]",
}


\definecolor{keywordcolor}{RGB}{0,0,0}
\definecolor{commentcolor}{RGB}{50,50,50}
\definecolor{opcolor}{RGB}{0,0,150}
\definecolor{literalcolor}{RGB}{0,0,150}
\definecolor{todefine}{RGB}{205,92,150}
\definecolor{typecolor}{RGB}{0,150,0}

\lstdefinestyle{pie}
{
    language=pie,
    basicstyle=\ttfamily,
    numbers=left,
    numberstyle=\tiny,
    keywordstyle=[1]\bfseries\color{keywordcolor},
    keywordstyle=[2]\bfseries\color{keywordcolor},
    keywordstyle=[3]\bfseries\color{keywordcolor},
    keywordstyle=[4]\bfseries\color{literalcolor},
    keywordstyle=[5]\bfseries\color{todefine},
    commentstyle=\color{commentcolor},
    showstringspaces=false
}

\lstset{style=pie}

%
 
\lstset{language=pie}

\usepackage{hyperref}
\hypersetup{
    colorlinks=true,
    linkcolor=black,
    urlcolor=black,
    linktoc=all
}

\usepackage{titlesec}

\titleformat{\chapter}
  {\Large\bfseries} % format
  {}                % label
  {0pt}             % sep
  {\huge}           % before-code



\title{DIM0661-PB1}
%\subtitle{Definição da linguagem e análise léxica}
\author{Grupo 3}
\date{\today}

\begin{document}

\maketitle

\tableofcontents

\chapter{Introdução}
Este relatório apresenta o manual da linguagem de programação que está sendo desenvolvida na disciplina de Compiladores (DIM0661). A linguagem deve obedecer as seguintes restrições:

\begin{changemargin}{1cm}{1cm}
\begin{itemize}
    \item deve ser parecida com Pascal (Pascal-like);
    \item deve ser em inglês;
    \item não deve possuir \texttt{while} e nem \texttt{repeat until};
    \item deve possuir um loop geral e permitir uma saída do loop (\texttt{exitwhen});
    \item deve ter tipagem fraca.
\end{itemize}
\end{changemargin}

O nome escolhido para a linguagem foi PIE, um acrônimo para Pascal-like (\textbf{P}ascal-l\textbf{I}k\textbf{E}).

\chapter{Regras sintáticas}
As regras sintáticas da linguagem foram construídas utilizando uma gramática livre de contexto que utiliza o formalismo de Backus-Naur (BNF).

\section{Gramática}

\begin{footnotesize}
\begin{lstlisting}[frame=single, label={prog}, language=pie]
<prog> ::= program id `;' <decl> <block> `.'
\end{lstlisting}

\begin{lstlisting}[frame=single, label={decl}, language=pie]
<decl> ::= <consts> <usertypes> <vars> <subprograms>
\end{lstlisting}

\begin{lstlisting}[frame=single, label={consts}, language=pie]
<consts> ::= `'                    |
             const <listconst> `;'
\end{lstlisting}

\begin{lstlisting}[frame=single, label={listconst}, language=pie]
<listconst> ::=  <constdecl> <listconstprime>
\end{lstlisting}

\begin{lstlisting}[frame=single, label={listconstprime}, language=pie]
<listconstprime> ::=  `'  |
		    <listconst>
\end{lstlisting}

\begin{lstlisting}[frame=single, label={constdecl}, language=pie]
<constdecl> ::= id `=' <expr> `; '
\end{lstlisting}

\begin{lstlisting}[frame=single, label={types}, language=pie]
<types> ::= id <typesprime> |
           <primtypes>
\end{lstlisting}

\begin{lstlisting}[frame=single, label={typesprime}, language=pie]
<typesprime> ::= `'                  |
                 `..' <subrangetype>

\end{lstlisting}

\begin{lstlisting}[frame=single, label={primtypes}, language=pie]
<primtypes> ::= int <typesprime>    |
                real                |
                bool                |
                char   <typesprime> |
                string              |
                <arraytype>         |
                <settype>           |
                <enumtype>          |
                <recordtype>
\end{lstlisting}

\begin{lstlisting}[frame=single, label={arraytype}, language=pie]
<arraytype> ::= array `[' <subrangelist> `]' of <types>
\end{lstlisting}

\begin{lstlisting}[frame=single, label={subrangelist}, language=pie]
<subrangelist> ::= id <subrangeprime>  |
                   int `..' <subrangetype> <subrangelistprime> |
                   char `..' <subrangetype> <subrangelistprime>
\end{lstlisting}

\begin{lstlisting}[frame=single, label={subrangeprime}, language=pie]
<subrangeprime> ::= <subrangelistprime>  |
                   `..' <subrangetype> <subrangelistprime>
\end{lstlisting}


\begin{lstlisting}[frame=single, label={subrangelistprime}, language=pie]
<subrangelistprime> ::=  `' |
	                 `,' <subrangelist>
\end{lstlisting}
\begin{lstlisting}[frame=single, label={subrangetype}, language=pie]
<subrangetype> ::= id     |
                   int    |
                   char
\end{lstlisting}

\begin{lstlisting}[frame=single, label={settype}, language=pie]
<settype> ::= set of <types>
\end{lstlisting}

\begin{lstlisting}[frame=single, label={enumtype}, language=pie]
<enumtype> ::= `(' <idlist> `)'
\end{lstlisting}

\begin{lstlisting}[frame=single, label={recordtype}, language=pie]
<recordtype> ::= record <varlistlist> end
\end{lstlisting}

\begin{lstlisting}[frame=single, label={usertypes}]
<usertypes> ::= `'                     |
                type <listusertypes>
\end{lstlisting}

\begin{lstlisting}[frame=single, label={listusertypes}]
<listusertypes> ::= <usertype> <listusertypesprime>
\end{lstlisting}

\begin{lstlisting}[frame=single, label={listusertypesprime}]
<listusertypesprime> ::= `'               |
                         <listusertypes>
\end{lstlisting}

\begin{lstlisting}[frame=single, label={usertypes}]
<usertype> ::= id `=` <types> `;'
\end{lstlisting}

\begin{lstlisting}[frame=single, label={vars}, language=pie]
<vars> ::= `'                |
           var <varlistlist>
\end{lstlisting}

\begin{lstlisting}[frame=single, label={varlistlist}, language=pie]
<varlistlist> ::= <varlist> <varlistlistprime>
\end{lstlisting}

\begin{lstlisting}[frame=single, label={varlistlistprime}, language=pie]
<varlistlistprime> ::= `'            |
                      <varlistlist>
\end{lstlisting}

\begin{lstlisting}[frame=single, label={varlist}, language=pie]
<varlist> ::= <types> <idlist> `;'
\end{lstlisting}

\begin{lstlisting}[frame=single, label={idlist}, language=pie]
<idlist> ::= id <idattr> <idlistprime>
\end{lstlisting}

\begin{lstlisting}[frame=single, label={idlistprime}, language=pie]
<idlistprime> ::= `'          |
                  `,' <idlist>
\end{lstlisting}

\begin{lstlisting}[frame=single, label={idattr}, language=pie]
<idattr> ::= `'         |
             `=' <expr>
\end{lstlisting}

\begin{lstlisting}[frame=single, label={variable}, language=pie]
<variable> ::= `->' id  <variableprime>                 |
               `[' <exprlistplus> `]' < variableprime>
\end{lstlisting}

\begin{lstlisting}[frame=single, label={variableprime}, language=pie]
<variableprime> ::= `'         |
                    <variable>
\end{lstlisting}

\begin{lstlisting}[frame=single, label={block}, language=pie]
<block> ::= begin <stmts> end
\end{lstlisting}

\begin{lstlisting}[frame=single, label={stmts}, language=pie]
<stmts> ::=  <stmt> <stmtlistprime>
\end{lstlisting}

\begin{lstlisting}[frame=single, label={stmtlistprime}, language=pie]
<stmtlistprime> ::= `' |
                    `;' <stmts>
\end{lstlisting}

\begin{lstlisting}[frame=single, label={stmt}, language=pie]
<stmt> ::= `'             |
           label <stmt>   |
           <block>        |
           <writestmt>    |
           <writelnstmt>  |
           <readstmt>     |
           <readlnstmt>   |
           <loopblock>    |
           <ifblock>      |
           <forblock>     |
           <caseblock>    |
           <gotostmt>     |
           <exitstmt>     |
           <returnstmt>   |
           id <stmtprime>
\end{lstlisting}

\begin{lstlisting}[frame=single, label={stmtprime}, language=pie]
<stmtprime> ::= <attrstmt> |
                <subprogcall>
\end{lstlisting}

\begin{lstlisting}[frame=single, label={subprogcall}, language=pie]
<subprogcall> ::= `(' <exprlist> `)'
\end{lstlisting}

\begin{lstlisting}[frame=single, label={exitstmt}, language=pie]
<exitstmt> ::= exitwhen <expr>
\end{lstlisting}

\begin{lstlisting}[frame=single, label={returnstmt}, language=pie]
<returnstmt> ::= return <expr>
\end{lstlisting}

\begin{lstlisting}[frame=single, label={attrstmt}, language=pie]
<attrstmt> ::= <variable> `:=' <expr> |
               `:=' <expr>
\end{lstlisting}

\begin{lstlisting}[frame=single, label={ifblock}, language=pie]
<ifblock> ::= if <expr> <stmt> <elseblock>
\end{lstlisting}


\begin{lstlisting}[frame=single, label={elseblock}, language=pie]
<elseblock> ::= `'          |
                else <stmt>
\end{lstlisting}

\begin{lstlisting}[frame=single, label={loopblock}, language=pie]
<loopblock> ::= loop <stmt>
\end{lstlisting}

\begin{lstlisting}[frame=single, label={caseblock}, language=pie]
<caseblock> ::= case <expr> of <caselist> <caseblockprime>
\end{lstlisting}

\begin{lstlisting}[frame=single, label={caseblockprime}, language=pie]
<caseblockprime> ::= end |
                     else <stmt> end
\end{lstlisting}

\begin{lstlisting}[frame=single, label={caselist}, language=pie]
<caselist> ::= <literallist> `:' <stmt> `;'
\end{lstlisting}

\begin{lstlisting}[frame=single, label={literallist}, language=pie]
<literallist> ::= <literal> <literallistprime>
\end{lstlisting}

\begin{lstlisting}[frame=single, label={literallistprime}, language=pie]
<literallistprime> ::= `' |
                       `,' <literallist>
\end{lstlisting}

\begin{lstlisting}[frame=single, label={gotostmt}, language=pie]
<gotostmt> ::= goto label
\end{lstlisting}

\begin{lstlisting}[frame=single, label={forblock}, language=pie]
<forblock> ::= for id <forblockprime>
\end{lstlisting}

\begin{lstlisting}[frame=single, label={forblockprime}, language=pie]
<forblockprime> ::= <variable> `:=' <expr> to <expr> step <expr> do <stmt> |
                   `:=' <expr> to <expr> step <expr> do <stmt>
\end{lstlisting}

\begin{lstlisting}[frame=single, label={expr}, language=pie]
<expr> ::= <conj>  <disj>
\end{lstlisting}

\begin{lstlisting}[frame=single, label={disj}, language=pie]
<disj> ::= `' |
           `||' <conj>
\end{lstlisting}

\begin{lstlisting}[frame=single, label={conj}, language=pie]
<conj> ::= <comp> <conjprime>
\end{lstlisting}

\begin{lstlisting}[frame=single, label={conjprime}, language=pie]
<conjprime> ::=  `'           |
                 `&&' <comp>
\end{lstlisting}

\begin{lstlisting}[frame=single, label={comp}, language=pie]
<comp> ::= <relational> <comprime>
\end{lstlisting}

\begin{lstlisting}[frame=single, label={relational}, language=pie]
<relational> ::= <sum> <relationalprime>
\end{lstlisting}

\begin{lstlisting}[frame=single, label={sum}, language=pie]
<sum> ::= <neg> <sumprime>
\end{lstlisting}

\begin{lstlisting}[frame=single, label={sumprime}, language=pie]
<sumprime> ::= `'                       |
              <add_op> <neg> <sumprime>
\end{lstlisting}

\begin{lstlisting}[frame=single, label={neg}, language=pie]
<neg> ::= <mul>         |
          `!' <mul>
\end{lstlisting}

\begin{lstlisting}[frame=single, label={mul}, language=pie]
<mul> ::= <final_term> <mulprime>
\end{lstlisting}

\begin{lstlisting}[frame=single, label={mulprime}, language=pie]
<mulprime> ::= `'                              |
                <mul_op> <final_term> <mulprime>
\end{lstlisting}

\begin{lstlisting}[frame=single, label={relationalprime}, language=pie]
<relationalprime> ::= `'                    |
                      <relational_op> <sum>
\end{lstlisting}

\begin{lstlisting}[frame=single, label={comprime}, language=pie]
<comprime> ::= `' |
              <equality_op> <relational>
\end{lstlisting}

\begin{lstlisting}[frame=single, label={final_term}, language=pie]
<final_term> ::= id <final_termprime> |
                <literal>             |
                `(' <expr> `)'
\end{lstlisting}

\begin{lstlisting}[frame=single, label={final_termprime}, language=pie]
<final_termprime> ::= `'         |
                    <variable>   |
                    <subprogcall>
\end{lstlisting}

\begin{lstlisting}[frame=single, label={add_op}, language=pie]
<add_op> ::= `+' |
             `-'
\end{lstlisting}

\begin{lstlisting}[frame=single, label={mul_op}, language=pie]
<mul_op> ::= `*' |
             `/' |
             `%'
\end{lstlisting}

\begin{lstlisting}[frame=single, label={equality_op}, language=pie]
<equality_op> ::= `==' |
                  `!='
\end{lstlisting}

\begin{lstlisting}[frame=single, label={relational_op}, language=pie]
<relational_op> ::= `<' |
	   	    `<=' |
		    `>'  |
		    `>='
\end{lstlisting}

\begin{lstlisting}[frame=single, label={literal}, language=pie]
<literal> ::= intliteral        |
              realiteral        |
              charliteral       |
              stringliteral     |
              <subrangeliteral>
\end{lstlisting}

\begin{lstlisting}[frame=single, label={exprlist}, language=pie]
<exprlist> ::= `' |
               <exprlistplus>
\end{lstlisting}

\begin{lstlisting}[frame=single, label={exprlistplus}, language=pie]
<exprlistplus> ::= <expr> <exprlistplusprime>
\end{lstlisting}

\begin{lstlisting}[frame=single, label={exprlistplusprime}, language=pie]
<exprlistplusprime> ::= `'              |
                     `,' < exprlistplus>
\end{lstlisting}

\begin{lstlisting}[frame=single, label={subprograms}, language=pie]
<subprograms> ::= `'                               |
                  <procedure> <subprogramsprime>   |
                  <function>  <subprogramsprime>
\end{lstlisting}

\begin{lstlisting}[frame=single, label={subprogramsprime}, language=pie]
<subprogramsprime> ::= `'                |
                       `;' <subprograms>
\end{lstlisting}

\begin{lstlisting}[frame=single, label={procedure}, language=pie]
<procedure> ::= proc id `(' <param> `)' `;' <decl> <block>
\end{lstlisting}

\begin{lstlisting}[frame=single, label={function}, language=pie]
<function> ::= func <types> id `(' <param> `)' `;' <decl> <block>
\end{lstlisting}

\begin{lstlisting}[frame=single, label={param}, language=pie]
<param> ::= `'            |
            <paramlistlist>
\end{lstlisting}

\begin{lstlisting}[frame=single, label={paramlistlist}, language=pie]
<paramlistlist> ::= `' |
                    `;' <paramlistlist>
\end{lstlisting}

\begin{lstlisting}[frame=single, label={paramlistlist}, language=pie]
<paramlist> ::= <types> <idlist> | 
                ref <types> <idlist>
\end{lstlisting}

\begin{lstlisting}[frame=single, label={paramlistlistprime}, language=pie]
<paramlistlistprime> ::= <paramlist> <paramlistlistprime>
\end{lstlisting}

\begin{lstlisting}[frame=single, label={writestmt}, language=pie]
<writestmt> ::= write `(' <expr> `)'
\end{lstlisting}

\begin{lstlisting}[frame=single, label={writelnstmt}, language=pie]
<writelnstmt> ::= writeln `(' <expr> `)'
\end{lstlisting}

\begin{lstlisting}[frame=single, label={readstmt}, language=pie]
<readstmt> ::= read `(' id <variableprime> `)'
\end{lstlisting}

\begin{lstlisting}[frame=single, label={readlnstmt}, language=pie]
<readlnstmt> ::= readln `(' id <variableprime> `)'
\end{lstlisting}
\end{footnotesize}


\newpage
\chapter{Termos léxicos}
\begin{verbatim}
    ; ( ) [ ] { } nil program proc begin end func const type var if else goto of
    for to do step in loop exitwhen case write writeln read readln return
    int bool real char string array set record enum subrange
\end{verbatim}
\section{Operadores aritméticos (numpericop)}
\begin{verbatim}
    +  -  *  /  % 
\end{verbatim}

\section{Operadores de conjuntos}
\begin{Verbatim}[commandchars=\\\{\}]
    + - * = != <= in
\end{Verbatim}

\section{Operadores de declaração}
\begin{verbatim}
    =
\end{verbatim}

\section{Operadores de atribuição}
\begin{Verbatim}[commandchars=\\\{\}]
    :=
\end{Verbatim}

\section{Operadores de comparação (boolop)}
\begin{Verbatim}[commandchars=\\\{\}]
    >  <  >=  <=  == !=
\end{Verbatim}

\section{Operadores lógicos (boolop)}
\begin{Verbatim}[commandchars=\\\{\}]
    && || !
\end{Verbatim}

\section{Literais booleanos}
\begin{verbatim}
    true false
\end{verbatim}

\newpage
\chapter{Regras}
Meta-operadores para definir as expressões regulares:
\begin{verbatim}
    [ ] : para enumerações associadas à -
    *   : repetição
    +   : para repetições de um vez ou mais
    ?   : de zero a uma vez
    .   : como um caractere ``joker'' exceto \n
    ^   : para o complementar de um  [ ]
    |   : para representar uma alternativa
\end{verbatim}

\section{id}
Identificadores podem começar apenas com letras, podem ter números e underline (``\_'') em sua estrutura.
A expressão regular que gera um identificador correto é:
 \begin{verbatim}
   id :  [a-zA-Z][a-zA-Z0-9_]*
\end{verbatim}

\section{label}
 \begin{verbatim}
   label : "@"[a-zA-Z0-9_]*
\end{verbatim}

\section{char}
\begin{verbatim}
   charliteral : \`[^']*\'
\end{verbatim}

\section{string}
\begin{verbatim}
   stringliteral : \"[^"\n]*\"
\end{verbatim}

\section{int}
\begin{verbatim}
   intliteral :  (("-"|"+")?[0-9]+)
\end{verbatim}

\section{real}
\begin{verbatim}
   exponent : ([E|e]("+"|"-")?({DIGIT}+))
   real : ([0-9]*[.])?[0-9]+
   realexponent : ([0-9]*[.])?[0-9]+{exponent}?
   realliteral :  (("-"|"+")?{real}|("-"|"+")?{realexponent})
\end{verbatim}

\section{Precedência}
A ordem de precedência deve valer para os seguintes operadores (), [], \{\}, $*$, $/$, \%, !, +, -, $<$, $<=$, $>$, $>=$, $:=$, $=$, $==$, $!=$, $\&\&$, $||$. A ordem de precedência pode ser visualizada na Tabela \ref{table:ordem_precedencia}.

\begin{table*}[h]
\renewcommand{\arraystretch}{1.34}
\centering
\begin{tabular}{| c | c | c | c | c | c |}
\hline
\bfseries Operador & \bfseries Precedência  \\
\hline
() & maior \\ \hline
[] &  \\ \hline
\{\} &  \\ \hline
 *, /, \% & \\ \hline
  ! & \\ \hline
 +, - & \\ \hline
$<$, $<=$, $>$, $>=$ & \\ \hline
$==$, $!=$ & \\ \hline
\&\& & \\ \hline
$||$ & \\ \hline
$:=$, = & menor\\ \hline
\end{tabular}
\caption{Ordem de precedência para os operadores.}
\label{table:ordem_precedencia}
\end{table*}

% Pascal tem funções matemáticas na própria linguagem (vamos abortar ou trataremos como se fosse de uma biblioteca?)
\section{Regra para comentários}
    Comentários iniciam por ``\#'' e são eliminados no pré-processamento.
\begin{verbatim}
   linecomment : "#"((.)*)\n
\end{verbatim}


\newpage
\chapter{Forma de uso}
O código se encontra em \url{https://github.com/raquel-oliveira/PIE}
\section{Compilação}
\begin{verbatim}
    lex pie.l
    cc lex.yy.c -o name -ll
\end{verbatim}

\section{Execução}
\begin{verbatim}
    ./name
\end{verbatim}

ou

\begin{verbatim}
    ./name < pathtofile.pie
\end{verbatim}

\chapter{Exemplos}

\begin{verbatim}
    lex pie.l
    cc lex.yy.c -o pie -ll
    ./pie < codesamples/merge_sort.pie
\end{verbatim}

\newpage
%http://rextester.com/GHRH16649
\section{MergeSort}
\begin{footnotesize}
\lstinputlisting{../codesamples/merge_sort.pie}
\end{footnotesize}

%http://sandbox.mc.edu/~bennet/cs404/doc/qsort_pas.html
\section{Quicksort}
\begin{footnotesize}
\lstinputlisting{../codesamples/quick_sort.pie}
\end{footnotesize}

%http://rextester.com/UXP1971
\section{Fatorial}
\begin{footnotesize}
\lstinputlisting{../codesamples/factorial.pie}
\end{footnotesize}

\nocite{*}
\bibliographystyle{ieeetr}
\bibliography{bibpie}


\end{document}

\documentclass[12pt]{report}
\usepackage[utf8]{inputenc}
\usepackage[brazilian]{babel}
%\usepackage[protrusion=true,expansion=true]{microtype} % S2  %
\usepackage{listings}
\usepackage{fullpage} % changes the margin
\usepackage{pxfonts} % to bold in listings
\usepackage{xcolor}
\usepackage{fancyvrb}
\usepackage{float}
\usepackage{txfonts}
\let\iint\relax
\let\iiint\relax
\let\iiiint\relax
\let\idotsint\relax
\usepackage{amsmath}
\usepackage{comment}

\def\changemargin#1#2{\list{}{\rightmargin#2\leftmargin#1}\item[]}
\let\endchangemargin=\endlist

\usepackage{titlesec}
\usepackage{longtable}
\usepackage{multirow}
\usepackage{array}
\newcolumntype{L}[1]{>{\raggedright\let\newline\\\arraybackslash\hspace{0pt}}m{#1}}
\newcolumntype{C}[1]{>{\centering\let\newline\\\arraybackslash\hspace{0pt}}m{#1}}
\newcolumntype{R}[1]{>{\raggedleft\let\newline\\\arraybackslash\hspace{0pt}}m{#1}}

%definition of language PIE
\lstdefinelanguage{pie}
{
    morekeywords={[1]
        program,
        proc,
        begin,
        end,
        func,
        const,
        type,
        var
    },
    morekeywords={[2]
        if,
        else,
        goto,
        of, %array, set, subrange
        for,
        to, %for
        do, %for
        step, %for
        in, %subrange, set
        loop,
        exitwhen, %loop
        case,
        write,
        writeln,
        read,
        readln,
        return,
        ref
    },
    morekeywords={[3]
        int,
        bool,
        real,
        char,
        string,
        array,
        set,
        record,
        enum,
        subrange
    },
    morekeywords={[4]
        true, false, nil
    },
    morekeywords={[5]
        id,
        label,
        intliteral,
        realiteral,
        charliteral,
        stringliteral
    },
    sensitive=true,
    morecomment=[l]{\#},
    morestring=[b]",
}


\definecolor{keywordcolor}{RGB}{0,0,0}
\definecolor{commentcolor}{RGB}{50,50,50}
\definecolor{opcolor}{RGB}{0,0,150}
\definecolor{literalcolor}{RGB}{0,0,150}
\definecolor{todefine}{RGB}{205,92,150}
\definecolor{typecolor}{RGB}{0,150,0}

\lstdefinestyle{pie}
{
    language=pie,
    basicstyle=\ttfamily,
    numbers=left,
    numberstyle=\tiny,
    keywordstyle=[1]\bfseries\color{keywordcolor},
    keywordstyle=[2]\bfseries\color{keywordcolor},
    keywordstyle=[3]\bfseries\color{keywordcolor},
    keywordstyle=[4]\bfseries\color{literalcolor},
    keywordstyle=[5]\bfseries\color{todefine},
    commentstyle=\color{commentcolor},
    showstringspaces=false
}

\lstset{style=pie}

%
 
\lstset{language=pie}

\usepackage{hyperref}
\hypersetup{
    colorlinks=true,
    linkcolor=black,
    urlcolor=black,
    linktoc=all
}

\usepackage{titlesec}

\titleformat{\chapter}
  {\Large\bfseries} % format
  {}                % label
  {0pt}             % sep
  {\huge}           % before-code



\title{DIM0661-PB2}
%\subtitle{Definição da linguagem e análise léxica}
\author{Grupo 3}
\date{\today}

\begin{document}

\maketitle

\tableofcontents

\chapter{Introdução}
Este relatório apresenta o manual da linguagem de programação que está sendo desenvolvida na disciplina de Compiladores (DIM0661). A linguagem deve obedecer as seguintes restrições:

\begin{changemargin}{1cm}{1cm}
\begin{itemize}
    \item deve ser parecida com Pascal (Pascal-like);
    \item deve ser em inglês;
    \item não deve possuir \texttt{while} e nem \texttt{repeat until};
    \item deve possuir um loop geral e permitir uma saída do loop (\texttt{exitwhen});
    \item deve ter tipagem fraca.
\end{itemize}
\end{changemargin}

O nome escolhido para a linguagem foi PIE, um acrônimo para Pascal-like (\textbf{P}ascal-l\textbf{I}k\textbf{E}).

\chapter{Regras sintáticas}
As regras sintáticas da linguagem foram construídas utilizando uma gramática livre de contexto que utiliza o formalismo de Backus-Naur (BNF).

\section{Gramática}

\begin{footnotesize}
\begin{lstlisting}[frame=single, label={prog}, language=pie]
<prog> ::= program id `;' <decl> <block> `.'
\end{lstlisting}

\begin{lstlisting}[frame=single, label={decl}, language=pie]
<decl> ::= <consts> <usertypes> <vars> <subprograms>
\end{lstlisting}

\begin{lstlisting}[frame=single, label={consts}, language=pie]
<consts> ::= `'                    |
             const <listconst> `;'
\end{lstlisting}

\begin{lstlisting}[frame=single, label={listconst}, language=pie]
<listconst> ::=  <constdecl> <listconstprime>
\end{lstlisting}

\begin{lstlisting}[frame=single, label={listconstprime}, language=pie]
<listconstprime> ::=  `'  |
		    <listconst>
\end{lstlisting}

\begin{lstlisting}[frame=single, label={constdecl}, language=pie]
<constdecl> ::= id `=' <expr> `; '
\end{lstlisting}

\begin{lstlisting}[frame=single, label={types}, language=pie]
<types> ::= id <typesprime> |
           <primtypes>
\end{lstlisting}

\begin{lstlisting}[frame=single, label={typesprime}, language=pie]
<typesprime> ::= `'                  |
                 `..' <subrangetype>

\end{lstlisting}

\begin{lstlisting}[frame=single, label={primtypes}, language=pie]
<primtypes> ::= int <typesprime>    |
                real                |
                bool                |
                char   <typesprime> |
                string              |
                <arraytype>         |
                <settype>           |
                <enumtype>          |
                <recordtype>
\end{lstlisting}

\begin{lstlisting}[frame=single, label={arraytype}, language=pie]
<arraytype> ::= array `[' <subrangelist> `]' of <types>
\end{lstlisting}

\begin{lstlisting}[frame=single, label={subrangelist}, language=pie]
<subrangelist> ::= id <subrangeprime>  |
                   int `..' <subrangetype> <subrangelistprime> |
                   char `..' <subrangetype> <subrangelistprime>
\end{lstlisting}

\begin{lstlisting}[frame=single, label={subrangeprime}, language=pie]
<subrangeprime> ::= <subrangelistprime>  |
                   `..' <subrangetype> <subrangelistprime>
\end{lstlisting}


\begin{lstlisting}[frame=single, label={subrangelistprime}, language=pie]
<subrangelistprime> ::=  `' |
	                 `,' <subrangelist>
\end{lstlisting}
\begin{lstlisting}[frame=single, label={subrangetype}, language=pie]
<subrangetype> ::= id     |
                   int    |
                   char
\end{lstlisting}

\begin{lstlisting}[frame=single, label={settype}, language=pie]
<settype> ::= set of <types>
\end{lstlisting}

\begin{lstlisting}[frame=single, label={enumtype}, language=pie]
<enumtype> ::= `(' <idlist> `)'
\end{lstlisting}

\begin{lstlisting}[frame=single, label={recordtype}, language=pie]
<recordtype> ::= record <varlistlist> end
\end{lstlisting}

\begin{lstlisting}[frame=single, label={usertypes}]
<usertypes> ::= `'                     |
                type <listusertypes>
\end{lstlisting}

\begin{lstlisting}[frame=single, label={listusertypes}]
<listusertypes> ::= <usertype> <listusertypesprime>
\end{lstlisting}

\begin{lstlisting}[frame=single, label={listusertypesprime}]
<listusertypesprime> ::= `'               |
                         <listusertypes>
\end{lstlisting}

\begin{lstlisting}[frame=single, label={usertypes}]
<usertype> ::= id `=` <types> `;'
\end{lstlisting}

\begin{lstlisting}[frame=single, label={vars}, language=pie]
<vars> ::= `'                |
           var <varlistlist>
\end{lstlisting}

\begin{lstlisting}[frame=single, label={varlistlist}, language=pie]
<varlistlist> ::= <varlist> <varlistlistprime>
\end{lstlisting}

\begin{lstlisting}[frame=single, label={varlistlistprime}, language=pie]
<varlistlistprime> ::= `'            |
                      <varlistlist>
\end{lstlisting}

\begin{lstlisting}[frame=single, label={varlist}, language=pie]
<varlist> ::= <types> <idlist> `;'
\end{lstlisting}

\begin{lstlisting}[frame=single, label={idlist}, language=pie]
<idlist> ::= id <idattr> <idlistprime>
\end{lstlisting}

\begin{lstlisting}[frame=single, label={idlistprime}, language=pie]
<idlistprime> ::= `'          |
                  `,' <idlist>
\end{lstlisting}

\begin{lstlisting}[frame=single, label={idattr}, language=pie]
<idattr> ::= `'         |
             `=' <expr>
\end{lstlisting}

\begin{lstlisting}[frame=single, label={variable}, language=pie]
<variable> ::= `->' id  <variableprime>                 |
               `[' <exprlistplus> `]' < variableprime>
\end{lstlisting}

\begin{lstlisting}[frame=single, label={variableprime}, language=pie]
<variableprime> ::= `'         |
                    <variable>
\end{lstlisting}

\begin{lstlisting}[frame=single, label={block}, language=pie]
<block> ::= begin <stmts> end
\end{lstlisting}

\begin{lstlisting}[frame=single, label={stmts}, language=pie]
<stmts> ::=  <stmt> <stmtlistprime>
\end{lstlisting}

\begin{lstlisting}[frame=single, label={stmtlistprime}, language=pie]
<stmtlistprime> ::= `' |
                    `;' <stmts>
\end{lstlisting}

\begin{lstlisting}[frame=single, label={stmt}, language=pie]
<stmt> ::= `'             |
           label <stmt>   |
           <block>        |
           <writestmt>    |
           <writelnstmt>  |
           <readstmt>     |
           <readlnstmt>   |
           <loopblock>    |
           <ifblock>      |
           <forblock>     |
           <caseblock>    |
           <gotostmt>     |
           <exitstmt>     |
           <returnstmt>   |
           id <stmtprime>
\end{lstlisting}

\begin{lstlisting}[frame=single, label={stmtprime}, language=pie]
<stmtprime> ::= <attrstmt> |
                <subprogcall>
\end{lstlisting}

\begin{lstlisting}[frame=single, label={subprogcall}, language=pie]
<subprogcall> ::= `(' <exprlist> `)'
\end{lstlisting}

\begin{lstlisting}[frame=single, label={exitstmt}, language=pie]
<exitstmt> ::= exitwhen <expr>
\end{lstlisting}

\begin{lstlisting}[frame=single, label={returnstmt}, language=pie]
<returnstmt> ::= return <expr>
\end{lstlisting}

\begin{lstlisting}[frame=single, label={attrstmt}, language=pie]
<attrstmt> ::= <variable> `:=' <expr> |
               `:=' <expr>
\end{lstlisting}

\begin{lstlisting}[frame=single, label={ifblock}, language=pie]
<ifblock> ::= if <expr> <stmt> <elseblock>
\end{lstlisting}


\begin{lstlisting}[frame=single, label={elseblock}, language=pie]
<elseblock> ::= `'          |
                else <stmt>
\end{lstlisting}

\begin{lstlisting}[frame=single, label={loopblock}, language=pie]
<loopblock> ::= loop <stmt>
\end{lstlisting}

\begin{lstlisting}[frame=single, label={caseblock}, language=pie]
<caseblock> ::= case <expr> of <caselist> <caseblockprime>
\end{lstlisting}

\begin{lstlisting}[frame=single, label={caseblockprime}, language=pie]
<caseblockprime> ::= end |
                     else <stmt> end
\end{lstlisting}

\begin{lstlisting}[frame=single, label={caselist}, language=pie]
<caselist> ::= <literallist> `:' <stmt> `;'
\end{lstlisting}

\begin{lstlisting}[frame=single, label={literallist}, language=pie]
<literallist> ::= <literal> <literallistprime>
\end{lstlisting}

\begin{lstlisting}[frame=single, label={literallistprime}, language=pie]
<literallistprime> ::= `' |
                       `,' <literallist>
\end{lstlisting}

\begin{lstlisting}[frame=single, label={gotostmt}, language=pie]
<gotostmt> ::= goto label
\end{lstlisting}

\begin{lstlisting}[frame=single, label={forblock}, language=pie]
<forblock> ::= for id <forblockprime>
\end{lstlisting}

\begin{lstlisting}[frame=single, label={forblockprime}, language=pie]
<forblockprime> ::= <variable> `:=' <expr> to <expr> step <expr> do <stmt> |
                   `:=' <expr> to <expr> step <expr> do <stmt>
\end{lstlisting}

\begin{lstlisting}[frame=single, label={expr}, language=pie]
<expr> ::= <conj>  <disj>
\end{lstlisting}

\begin{lstlisting}[frame=single, label={disj}, language=pie]
<disj> ::= `' |
           `||' <conj>
\end{lstlisting}

\begin{lstlisting}[frame=single, label={conj}, language=pie]
<conj> ::= <comp> <conjprime>
\end{lstlisting}

\begin{lstlisting}[frame=single, label={conjprime}, language=pie]
<conjprime> ::=  `'           |
                 `&&' <comp>
\end{lstlisting}

\begin{lstlisting}[frame=single, label={comp}, language=pie]
<comp> ::= <relational> <comprime>
\end{lstlisting}

\begin{lstlisting}[frame=single, label={relational}, language=pie]
<relational> ::= <sum> <relationalprime>
\end{lstlisting}

\begin{lstlisting}[frame=single, label={sum}, language=pie]
<sum> ::= <neg> <sumprime>
\end{lstlisting}

\begin{lstlisting}[frame=single, label={sumprime}, language=pie]
<sumprime> ::= `'                       |
              <add_op> <neg> <sumprime>
\end{lstlisting}

\begin{lstlisting}[frame=single, label={neg}, language=pie]
<neg> ::= <mul>         |
          `!' <mul>
\end{lstlisting}

\begin{lstlisting}[frame=single, label={mul}, language=pie]
<mul> ::= <final_term> <mulprime>
\end{lstlisting}

\begin{lstlisting}[frame=single, label={mulprime}, language=pie]
<mulprime> ::= `'                              |
                <mul_op> <final_term> <mulprime>
\end{lstlisting}

\begin{lstlisting}[frame=single, label={relationalprime}, language=pie]
<relationalprime> ::= `'                    |
                      <relational_op> <sum>
\end{lstlisting}

\begin{lstlisting}[frame=single, label={comprime}, language=pie]
<comprime> ::= `' |
              <equality_op> <relational>
\end{lstlisting}

\begin{lstlisting}[frame=single, label={final_term}, language=pie]
<final_term> ::= id <final_termprime> |
                <literal>             |
                `(' <expr> `)'
\end{lstlisting}

\begin{lstlisting}[frame=single, label={final_termprime}, language=pie]
<final_termprime> ::= `'         |
                    <variable>   |
                    <subprogcall>
\end{lstlisting}

\begin{lstlisting}[frame=single, label={add_op}, language=pie]
<add_op> ::= `+' |
             `-'
\end{lstlisting}

\begin{lstlisting}[frame=single, label={mul_op}, language=pie]
<mul_op> ::= `*' |
             `/' |
             `%'
\end{lstlisting}

\begin{lstlisting}[frame=single, label={equality_op}, language=pie]
<equality_op> ::= `==' |
                  `!='
\end{lstlisting}

\begin{lstlisting}[frame=single, label={relational_op}, language=pie]
<relational_op> ::= `<' |
	   	    `<=' |
		    `>'  |
		    `>='
\end{lstlisting}

\begin{lstlisting}[frame=single, label={literal}, language=pie]
<literal> ::= intliteral        |
              realiteral        |
              charliteral       |
              stringliteral     |
              <subrangeliteral>
\end{lstlisting}

\begin{lstlisting}[frame=single, label={exprlist}, language=pie]
<exprlist> ::= `' |
               <exprlistplus>
\end{lstlisting}

\begin{lstlisting}[frame=single, label={exprlistplus}, language=pie]
<exprlistplus> ::= <expr> <exprlistplusprime>
\end{lstlisting}

\begin{lstlisting}[frame=single, label={exprlistplusprime}, language=pie]
<exprlistplusprime> ::= `'              |
                     `,' < exprlistplus>
\end{lstlisting}

\begin{lstlisting}[frame=single, label={subprograms}, language=pie]
<subprograms> ::= `'                               |
                  <procedure> <subprogramsprime>   |
                  <function>  <subprogramsprime>
\end{lstlisting}

\begin{lstlisting}[frame=single, label={subprogramsprime}, language=pie]
<subprogramsprime> ::= `'                |
                       `;' <subprograms>
\end{lstlisting}

\begin{lstlisting}[frame=single, label={procedure}, language=pie]
<procedure> ::= proc id `(' <param> `)' `;' <decl> <block>
\end{lstlisting}

\begin{lstlisting}[frame=single, label={function}, language=pie]
<function> ::= func <types> id `(' <param> `)' `;' <decl> <block>
\end{lstlisting}

\begin{lstlisting}[frame=single, label={param}, language=pie]
<param> ::= `'            |
            <paramlistlist>
\end{lstlisting}

\begin{lstlisting}[frame=single, label={paramlistlist}, language=pie]
<paramlistlist> ::= `' |
                    `;' <paramlistlist>
\end{lstlisting}

\begin{lstlisting}[frame=single, label={paramlistlist}, language=pie]
<paramlist> ::= <types> <idlist> | 
                ref <types> <idlist>
\end{lstlisting}

\begin{lstlisting}[frame=single, label={paramlistlistprime}, language=pie]
<paramlistlistprime> ::= <paramlist> <paramlistlistprime>
\end{lstlisting}

\begin{lstlisting}[frame=single, label={writestmt}, language=pie]
<writestmt> ::= write `(' <expr> `)'
\end{lstlisting}

\begin{lstlisting}[frame=single, label={writelnstmt}, language=pie]
<writelnstmt> ::= writeln `(' <expr> `)'
\end{lstlisting}

\begin{lstlisting}[frame=single, label={readstmt}, language=pie]
<readstmt> ::= read `(' id <variableprime> `)'
\end{lstlisting}

\begin{lstlisting}[frame=single, label={readlnstmt}, language=pie]
<readlnstmt> ::= readln `(' id <variableprime> `)'
\end{lstlisting}
\end{footnotesize}


\newpage
\chapter{Termos léxicos}
\begin{verbatim}
    ; ( ) [ ] { } nil program proc begin end func const type var if else goto of
    for to do step in loop exitwhen case write writeln read readln return
    int bool real char string array set record enum subrange
\end{verbatim}
\section{Operadores aritméticos (numpericop)}
\begin{verbatim}
    +  -  *  /  % 
\end{verbatim}

\section{Operadores de conjuntos}
\begin{Verbatim}[commandchars=\\\{\}]
    + - * = != <= in
\end{Verbatim}

\section{Operadores de declaração}
\begin{verbatim}
    =
\end{verbatim}

\section{Operadores de atribuição}
\begin{Verbatim}[commandchars=\\\{\}]
    :=
\end{Verbatim}

\section{Operadores de comparação (boolop)}
\begin{Verbatim}[commandchars=\\\{\}]
    >  <  >=  <=  == !=
\end{Verbatim}

\section{Operadores lógicos (boolop)}
\begin{Verbatim}[commandchars=\\\{\}]
    && || !
\end{Verbatim}

\section{Literais booleanos}
\begin{verbatim}
    true false
\end{verbatim}

\newpage
\chapter{Regras sobre símbolos terminais}
Meta-operadores para definir as expressões regulares:
\begin{verbatim}
    [ ] : para enumerações associadas à -
    *   : repetição
    +   : para repetições de um vez ou mais
    ?   : de zero a uma vez
    .   : como um caractere ``joker'' exceto \n
    ^   : para o complementar de um  [ ]
    |   : para representar uma alternativa
\end{verbatim}

\section{id}
Identificadores podem começar apenas com letras, podem ter números e underline (``\_'') em sua estrutura.
A expressão regular que gera um identificador correto é:
 \begin{verbatim}
   id :  [a-zA-Z][a-zA-Z0-9_]*
\end{verbatim}

\section{label}
 \begin{verbatim}
   label : "@"[a-zA-Z0-9_]*
\end{verbatim}

\section{char}
\begin{verbatim}
   charliteral : \`[^']*\'
\end{verbatim}

\section{string}
\begin{verbatim}
   stringliteral : \"[^"\n]*\"
\end{verbatim}

\section{int}
\begin{verbatim}
   intliteral :  (("-"|"+")?[0-9]+)
\end{verbatim}

\section{real}
\begin{verbatim}
   exponent : ([E|e]("+"|"-")?({DIGIT}+))
   real : ([0-9]*[.])?[0-9]+
   realexponent : ([0-9]*[.])?[0-9]+{exponent}?
   realliteral :  (("-"|"+")?{real}|("-"|"+")?{realexponent})
\end{verbatim}

\section{Precedência}
A ordem de precedência deve valer para os seguintes operadores (), [], \{\}, $*$, $/$, \%, !, +, -, $<$, $<=$, $>$, $>=$, $:=$, $=$, $==$, $!=$, $\&\&$, $||$. A ordem de precedência pode ser visualizada na Tabela \ref{table:ordem_precedencia}.

\begin{table*}[h]
\renewcommand{\arraystretch}{1.34}
\centering
\begin{tabular}{| c | c | c | c | c | c |}
\hline
\bfseries Operador & \bfseries Precedência  \\
\hline
() & maior \\ \hline
[] &  \\ \hline
\{\} &  \\ \hline
 *, /, \% & \\ \hline
  ! & \\ \hline
 +, - & \\ \hline
$<$, $<=$, $>$, $>=$ & \\ \hline
$==$, $!=$ & \\ \hline
\&\& & \\ \hline
$||$ & \\ \hline
$:=$, = & menor\\ \hline
\end{tabular}
\caption{Ordem de precedência para os operadores.}
\label{table:ordem_precedencia}
\end{table*}

% Pascal tem funções matemáticas na própria linguagem (vamos abortar ou trataremos como se fosse de uma biblioteca?)
\section{Regra para comentários}
    Comentários iniciam por ``\#'' e são eliminados no pré-processamento.
\begin{verbatim}
   linecomment : "#"((.)*)\n
\end{verbatim}


\newpage
\chapter{Analisador Sintático (Parser)}
Os dois parsers implementados (recursivo e por tabela) são parsers recursivos. Isso significa que não utilizam de backtracking e portanto rodam em tempo linear (de acordo com o tamnho do input, a palavra sendo analisada). Isso é importante pois o tempo de compilação de programas deve ser o menor possível. O parsing é feito analisando as regras sintáticas da linguagem e um input (uma palavra), e é utilizado para determinar se essa palavra faz parte da linguagem. 

\section{Conjuntos Predict}
Cada regra sintática tem associada a si um conjunto de terminais chamado \textit{Predict}.

Seja $A$ um símbolo não-terminal, $S$ o símbolo não-terminal inicial, $\alpha$ e $\beta$ sequências de não-terminais e/ou terminais e $T$ o conjunto dos símbolos terminais,

\begin{align*} 
Predict(A \rightarrow \alpha) & = First(A\rightarrow \alpha) \cup Follow(A), \text{ if } A \Rightarrow^* \lambda \\
 & = First(A\rightarrow \alpha), \text{ otherwise}
\end{align*}

\begin{equation*}
First(A \rightarrow \alpha) = \{s \in T \mid \alpha \Rightarrow^* s\beta\}
\end{equation*}

\begin{equation*}
Follow(A) = \{s \in T \mid S \Rightarrow^* ...As...\}
\end{equation*}

O conjunto Predict de uma regra $\alpha$ basicamente nos diz quais símbolos terminais são candidatos a serem o próximo a aparecer se usarmos $\alpha$. Tanto no parser recursivo como o por tabela, a cada passo nós estamos analisando um símbolo não-terminal $A$ (começando pelo símbolo inicial) e um terminal $s$ (vindo do input). Para sabermos qual regra utilizar, basta olharmos as regras de $A$ e acharmos a que tem $s$ no seu conjunto Predict. Se $s$ não estiver no conjunto Predict de nenhuma regra de $A$, o input não é uma palavra da linguagem, e um erro deve ser gerado. Como nossa gramática é LL1 (ignorando o caso do else), isso significa que não há interseção nos conjuntos Predict de regras de um mesmo símbolo não-terminal. Ou seja, para cada não-terminal $A$ e terminal $s$, no máximo uma regra de $A$ terá $s$ no seu Predict, permitindo que o parsing seja realizado da maneira descrita nesse parágrafo com sucesso. Os conjuntos Predict de todas as regras da nossa gramática podem ser vistos na Tabela bla.

\section{Parser preditivo recursivo}
No parser recursivo, é utilizada a pilha de execução do programa para fazer o parsing top-down. Há uma funcão para cada não-terminal e o programa é inicilizado chamando a função do não-terminal inicial. Na função de cada não-terminal $A$, fazemos um switch/case utilizando o próximo símbolo terminal $t$ainda não \textit{matched} do input, onde cada case é a regra tal que $t$ está no Predict dessa regra: para cada não-terminal $B$ da regra, chamamos a função de $B$, e para cada terminal $s$ da regra, chamamos um \textit{matching} de $s$ e $t$ (que gera um erro se $s$ e $t$ forem diferentes e avança no input para o próximo símbolo). O caso default é onde $t$ não está no Predict de nenhuma regra de $A$, e gera um erro.  

\section{Parser preditivo por tabela}
No parser por tabela, utilizamos uma pilha explícita e uma tabela. No nosso código, a tabela é um std::map onde a chave é um par de um símbolo não-terminal e um símbolo terminal e o valor é um vetor de símbolos terminais e não-terminais (uma regra). Essa tabela é construída utilizando os conjuntos Predict; há uma entrada na tabela para a chave ($A$,$s$) se $s$ está no Predict de alguma regra de $A$, e o valor associado a essa chave é justamente essa regra. O programa é inicilizado empilhando o símbolo não-terminal inicial, e então fica em um loop enquanto a pilha não estiver vazia. A cada passo desse loop, olhamos o símbolo no topo da pilha. Se for um símbolo terminal $s$, fazemos o \textit{matching} com o próximo símbolo terminal $t$ ainda não \textit{matched} do input (que gera um erro se $s$ e $t$ forem diferentes e avança no input para o próximo símbolo). Se for um símbolo não-terminal $A$, olhamos na tabela a chave {$A$,$t$} e empilhamos a regra (o vetor) correspondente. Caso não haja um vetor para essa chave, um erro é gerado.

\section{Recuperação de erro}
O compilador não deve parar sua execução ao encontrar um erro sintático, mas sim notificar o erro e continuar o parsing. Desse modo, vários erros podem ser reportados com uma só compilação. Para isso, ao encontrar um erro, é preciso utilizar de alguma técnica de recuperação de erro para que o parsing possa continuar. No nosso caso, utilizamos diferentes técnicas de recuperação de erro. 

No parser recursivo, quando há um erro de \textit{matching} ou um erro de símbolo não esperado (quando não há regra cujo Predict contém o símbolo), esse erro é notificado e então avançamos no input para o próximo símbolo. Isso é uma boa técnica quando erros de digitação são frequentes, mas resulta em muitos outros erros se há uma palavra faltando no input. 

Já no parser por tabela, o erro de símbolo não esperado é recuperado de uma forma diferente: desempilhamos a pilha até encontrarmos um $;$ (ou até ela ficar vazia) e avançamos no input até encontrarmos um $;$ (ou até ele acabar). A intuição é que estamos continuando o parsing a partir do próximo $;$. O erro de \textit{matching} é recuperado da mesma forma que no recursivo.

\section{Saídas dos parsers}
Os parsers não tem saída quando não há erros, ou seja, quando o programa (o input) for uma palavra da linguagem. Para todos os nossos programas de exemplo (factorial.pie, quick\_sort.pie e merge\_sort.pie), os dois parsers não estão imprimindo nenhuma saída, o que significa que esses programas estão corretos em relação a linguagem.

Quando há um erro, ele é disponibilizado das seguintes formas:
\vspace{0.5cm}

\begin{verbatim}
ERROR: At row 48 column 24 
    Not expected symbol: ;
ERROR: At row 49 column 17 
    Not expected symbol: ID_TOKEN (with lexeme k)
ERROR: At row 49 column 19
    Expected: END_TOKEN
    Found: ATTR_TOKEN
\end{verbatim} 




\newpage
\chapter{Forma de uso}
O código se encontra em \url{https://github.com/raquel-oliveira/PIE}
\section{Compilação}
\begin{verbatim}
    make
\end{verbatim}

\section{Execução}
\subsection{Parser recursivo}
\begin{verbatim}
    ./recursive pathtofile.pie
\end{verbatim}

\subsection{Parser por tabela}
\begin{verbatim}
    ./table pathtofile.pie
\end{verbatim}

\section{Exemplo}
\begin{verbatim}
    make
    ./recursive codesamples/mergesort.pie
\end{verbatim}

\newpage
\chapter{Exemplos de programas}
%http://rextester.com/GHRH16649
\section{MergeSort}
\begin{footnotesize}
\lstinputlisting{../codesamples/merge_sort.pie}
\end{footnotesize}

%http://sandbox.mc.edu/~bennet/cs404/doc/qsort_pas.html
\section{Quicksort}
\begin{footnotesize}
\lstinputlisting{../codesamples/quick_sort.pie}
\end{footnotesize}

%http://rextester.com/UXP1971
\section{Fatorial}
\begin{footnotesize}
\lstinputlisting{../codesamples/factorial.pie}
\end{footnotesize}

\newpage
\chapter{Participação}
\begin{table}[h]
	\centering
	\label{my-label}
	\begin{tabular}{|l|l|r|}
		\hline
		\textbf{Matrícula} & \textbf{Aluno(a)} & \textbf{Participação (\%)} \\ \hline
		20180008183        & BRENO MAURICIO DE FREITAS VIANA & 20\%                       \\ \hline
		20170153995        & FERNANDA MENEZES PAES ISABEL    & 20\%                       \\ \hline
		2012912517         & LUIZ ARTHUR DE LIMA FREIRE      & 20\%                       \\ \hline
		20180008281        & RAQUEL LOPES DE OLIVEIRA        & 20\%                       \\ \hline
		20180008316        & VÍTOR DE GODEIRO MARQUES        & 20\%                       \\ \hline
		\multicolumn{2}{|l|}{\textbf{TOTAL}}            & 100\%                      \\ \hline
	\end{tabular}
\end{table}

\nocite{*}
\bibliographystyle{ieeetr}
\bibliography{bibpie}


\end{document}

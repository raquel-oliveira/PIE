\documentclass[12pt]{report}
\usepackage[utf8]{inputenc}
\usepackage[brazilian]{babel}
%\usepackage[protrusion=true,expansion=true]{microtype} % S2  %
\usepackage{listings}
\usepackage{fullpage} % changes the margin
\usepackage{pxfonts} % to bold in listings
\usepackage{xcolor}
\usepackage{fancyvrb}
\usepackage{float}
\usepackage{txfonts}
\let\iint\relax
\let\iiint\relax
\let\iiiint\relax
\let\idotsint\relax
\usepackage{amsmath}
\usepackage{comment}

\def\changemargin#1#2{\list{}{\rightmargin#2\leftmargin#1}\item[]}
\let\endchangemargin=\endlist

%definition of language PIE
\lstdefinelanguage{pie}
{
    morekeywords={[1]
        program,
        proc,
        begin,
        end,
        func,
        const,
        type,
        var
    },
    morekeywords={[2]
        if,
        else,
        goto,
        of, %array, set, subrange
        for,
        to, %for
        do, %for
        step, %for
        in, %subrange, set
        loop,
        exitwhen, %loop
        case,
        write,
        writeln,
        read,
        readln,
        return,
        ref
    },
    morekeywords={[3]
        int,
        bool,
        real,
        char,
        string,
        array,
        set,
        record,
        enum,
        subrange
    },
    morekeywords={[4]
        true, false, nil
    },
    morekeywords={[5]
        id,
        label,
        intliteral,
        realiteral,
        charliteral,
        stringliteral
    },
    sensitive=true,
    morecomment=[l]{\#},
    morestring=[b]",
}


\definecolor{keywordcolor}{RGB}{0,0,0}
\definecolor{commentcolor}{RGB}{50,50,50}
\definecolor{opcolor}{RGB}{0,0,150}
\definecolor{literalcolor}{RGB}{0,0,150}
\definecolor{todefine}{RGB}{205,92,150}
\definecolor{typecolor}{RGB}{0,150,0}

\lstdefinestyle{pie}
{
    language=pie,
    basicstyle=\ttfamily,
    numbers=left,
    numberstyle=\tiny,
    keywordstyle=[1]\bfseries\color{keywordcolor},
    keywordstyle=[2]\bfseries\color{keywordcolor},
    keywordstyle=[3]\bfseries\color{keywordcolor},
    keywordstyle=[4]\bfseries\color{literalcolor},
    keywordstyle=[5]\bfseries\color{todefine},
    commentstyle=\color{commentcolor},
    showstringspaces=false
}

\lstset{style=pie}

%
 
\lstset{language=pie}

\usepackage{hyperref}
\hypersetup{
    colorlinks=true,
    linkcolor=black,
    urlcolor=black,
    linktoc=all
}

\usepackage{titlesec}

\titleformat{\chapter}
  {\Large\bfseries} % format
  {}                % label
  {0pt}             % sep
  {\huge}           % before-code



\title{DIM0661-PB1}
%\subtitle{Definição da linguagem e análise léxica}
\author{Grupo 3}
\date{\today}

\begin{document}

\maketitle

\tableofcontents

\chapter{Introdução}
Este relatório apresenta o manual da linguagem de programação que está sendo desenvolvida na disciplina de Compiladores (DIM0661). A linguagem deve obedecer as seguintes restrições:

\begin{changemargin}{1cm}{1cm}
\begin{itemize}
    \item deve ser parecida com Pascal (Pascal-like);
    \item deve ser em inglês;
    \item não deve possuir \texttt{while} e nem \texttt{repeat until};
    \item deve possuir um loop geral e permitir uma saída do loop (\texttt{exitwhen});
    \item deve ter tipagem fraca.
\end{itemize}
\end{changemargin}

O nome escolhido para a linguagem foi PIE, um acrônimo para Pascal-like (\textbf{P}ascal-l\textbf{I}k\textbf{E}).

\chapter{Regras sintáticas}
As regras sintáticas da linguagem foram construídas utilizando uma gramática livre de contexto que utiliza o formalismo de Backus-Naur (BNF).


\section{Gramática}

\begin{footnotesize}
\begin{lstlisting}[frame=single, label={prog}, language=pie]
<prog> ::= program id `;' <decl> <block> `.'
\end{lstlisting}

\begin{lstlisting}[frame=single, label={decl}, language=pie]
<decl> ::= <consts> <usertypes> <vars> <subprograms>
\end{lstlisting}

\begin{lstlisting}[frame=single, label={consts}, language=pie]
<consts> ::= `'                    | 
             const <listconst> `;'
\end{lstlisting}

\begin{lstlisting}[frame=single, label={listconst}, language=pie]
<listconst> ::= <constdecl>                 |
                <constdecl> `;' <listconst>
\end{lstlisting}

\begin{lstlisting}[frame=single, label={constdecl}, language=pie]
<constdecl> ::= id `=' <expr>
\end{lstlisting}

\begin{lstlisting}[frame=single, label={types}, language=pie]
<types> ::= id        |
           <primtype>
\end{lstlisting}

\begin{lstlisting}[frame=single, label={primtypes}, language=pie]
<primtypes> ::= int            |
                real           | 
                bool           |
                char           |
                string         |
                <arraytype>    |
                <settype>      |
                <enumtype>     |
                <subrangetype> |
                <recordtype>
\end{lstlisting}

\begin{lstlisting}[frame=single, label={arraytype}, language=pie]
<arraytype> ::= array `[' <subrangelist> `]' of <types>
\end{lstlisting}

\begin{lstlisting}[frame=single, label={subrangelist}, language=pie]
<subrangelist> ::= <subrangetype>                    |
                   <subrangetype> `,' <subrangelist>
\end{lstlisting}

\begin{lstlisting}[frame=single, label={subrangetype}, language=pie]
<subrangetype> ::= id `..' id     |
                   int `..' int   | 
                   char `..' char
\end{lstlisting}

\begin{lstlisting}[frame=single, label={settype}, language=pie]
<settype> ::= set of <types>
\end{lstlisting}

\begin{lstlisting}[frame=single, label={enumtype}, language=pie]
<enumtype> ::= `(' <idlist> `)'
\end{lstlisting}

\begin{lstlisting}[frame=single, label={recordtype}, language=pie]
<recordtype> ::= record <varlistlist> end
\end{lstlisting}

\begin{lstlisting}[frame=single, label={usertypes}]
<usertypes> ::= `'                       |
                type <listusertypes> `;'
\end{lstlisting}

\begin{lstlisting}[frame=single, label={listusertypes}]
<listusertypes> ::= <usertype> | <usertype> `;' <listusertypes>
\end{lstlisting}

\begin{lstlisting}[frame=single, label={listusertypes}]
<usertype> ::= id `=` <types>
\end{lstlisting}

\begin{lstlisting}[frame=single, label={vars}, language=pie]
<vars> ::= `'                |
           var <varlistlist>
\end{lstlisting}

\begin{lstlisting}[frame=single, label={varlistlist}, language=pie]
<varlistlist> ::= <varlist>               |
                  <varlist> <varlistlist>
\end{lstlisting}

%mudar
\begin{lstlisting}[frame=single, label={varlist}, language=pie]
<varlist> ::= <types> <idlist> `;' |  <types> <idattrlist> `;'
\end{lstlisting}

\begin{lstlisting}[frame=single, label={idlist}, language=pie]
<idattrlist> ::= id = <expr>                  | 
                 id = <expr> `,' <idattrlist> 
\end{lstlisting}

\begin{lstlisting}[frame=single, label={idlist}, language=pie]
<idlist> ::= id              | 
             id `,' <idlist> 
\end{lstlisting}

\begin{lstlisting}[frame=single, label={variable}, language=pie]
<variable> ::= id                     | 
               <variable> `->' id     |
               id `[' <exprlist+> `]'
\end{lstlisting}

\begin{lstlisting}[frame=single]
<block> ::= begin <stmts> end
\end{lstlisting}

\begin{lstlisting}[frame=single]
<stmts> ::= `'         | 
            <stmtlist>
\end{lstlisting}

\begin{lstlisting}[frame=single]
<stmtlist> ::= <stmt>                | 
               <stmt> `;' <stmtlist>
\end{lstlisting}

\begin{lstlisting}[frame=single]
<stmt> ::= `'             | 
           label <stmt>   |
           <block>        |
           <writeblock>   |
           <writelnblock> |
           <readblock>    |
           <readlnblock>  |
           <loopblock>    |
           <ifblock>      |
           <forblock>     |
           <caseblock>    |
           <gotostmt>     |
           <attrstmt>     |
           <exitstmt>     |
           <returnstmt>   |
           <subprogcall>
\end{lstlisting}

\begin{lstlisting}[frame=single]
<subprogcall> ::= id `(' <exprlist> `)'
\end{lstlisting}

\begin{lstlisting}[frame=single]
<exitstmt> ::= exitwhen `(' <boolexpr> `)'
\end{lstlisting}

\begin{lstlisting}[frame=single]
<returnstmt> ::= return <expr>
\end{lstlisting}

\begin{lstlisting}[frame=single]
<attrstmt> ::= <variable> `:=' <expr>
\end{lstlisting}

\begin{lstlisting}[frame=single]
<ifblock> ::= if `(' <boolexpr> `)' <stmt> <elseblock> 
\end{lstlisting}


\begin{lstlisting}[frame=single]
<elseblock> ::= `'          |
                else <stmt>
\end{lstlisting}

\begin{lstlisting}[frame=single]
<loopblock> ::= loop <stmt>
\end{lstlisting}

\begin{lstlisting}[frame=single]
<caseblock> ::= case <expr> of <caselist> end             |
                case <expr> of <caselist> else <stmt> end
\end{lstlisting}

\begin{lstlisting}[frame=single]
<caselist> ::= <discretevalue> `:' <stmt> `;'      |
               <discretevalueslist> `:' <stmt> `;'
\end{lstlisting}

\begin{lstlisting}[frame=single]
<discreetvalueslist> ::= <discretevalue>                          |
                         <discretevalue> `,' <discretevalueslist>
\end{lstlisting}

\begin{lstlisting}[frame=single]
<discretevalue> ::= % TODO
\end{lstlisting}

\begin{lstlisting}[frame=single]
<gotostmt> ::= goto label
\end{lstlisting}

\begin{lstlisting}[frame=single]
<forblock> ::= for <variable> `:=' <expr> to <expr>
    step <expr> do <stmt>
\end{lstlisting}

\begin{lstlisting}[frame=single]
<expr> ::= <variable>         |
           literal            |
           <subprogcall>      |
           `(' <expr> `)'     |
           `!'<expr>          |
           `+'<expr>          |
           `-'<expr>          |
           <expr> `+' <expr>  |
           <expr> `-' <expr>  |
           <expr> `*' <expr>  |
           <expr> `/' <expr>  |
           <expr> `%' <expr>  |
           <expr> `&&' <expr> |
           <expr> `||' <expr> |
           <expr> `>' <expr>  |
           <expr> `<' <expr>  |
           <expr> `>=' <expr> |
           <expr> `<=' <expr> |
           <expr> `==' <expr> |
           <expr> `!=' <expr>
\end{lstlisting}

\begin{lstlisting}[frame=single]
<literal> ::= intliteral        | 
              realiteral        | 
              charliteral       |
              stringliteral     |
              <subrangetype>
\end{lstlisting}

\begin{lstlisting}[frame=single]
<exprlist> ::= `' | <exprlist+>
\end{lstlisting}

\begin{lstlisting}[frame=single]
<exprlist+> ::= <expr> | <expr> `,' <exprlist+>
\end{lstlisting}

\begin{lstlisting}[frame=single]
<subprograms> ::= `'                            |
                  <procedure>                   |
                  <function>                    |
                  <procedure> `;' <subprograms> |
                  <function> `;' <subprograms>
\end{lstlisting}

\begin{lstlisting}[frame=single]
<procedure> ::= proc id `(' <param> `)' `;' <decl> <block> 
\end{lstlisting}

\begin{lstlisting}[frame=single]
<function> ::= func <types> id `(' <param> `)' `;' <decl> <block>
\end{lstlisting}

\begin{lstlisting}[frame=single]
<param> ::= `'            |
            <varlistlist>
\end{lstlisting}

\begin{lstlisting}[frame=single]
<writeblock> ::= write `(' <expr> `)'
\end{lstlisting}

\begin{lstlisting}[frame=single]
<writelnblock> ::= writeln `(' <expr> `)'
\end{lstlisting}

\begin{lstlisting}[frame=single]
<readblock> ::= read `(' id `)'
\end{lstlisting}

\begin{lstlisting}[frame=single]
<readlnblock> ::= readln `(' id `)'
\end{lstlisting}
\end{footnotesize}

\newpage
\chapter{Termos léxicos}
\begin{verbatim}
    ; ( ) [ ] { } nil program proc begin end func const type var if else goto of
    for to do step in loop exitwhen case write writeln read readln return ref
    int bool real char string array set record
\end{verbatim}
\section{Operadores aritméticos (numpericop)}
\begin{verbatim}
    +  -  *  /  % 
\end{verbatim}

\section{Operadores de conjuntos}
\begin{verbatim}
    + - * = != <= in
\end{verbatim}

\section{Operadores de declaração}
\begin{verbatim}
    =
\end{verbatim}

\section{Operadores de atribuição}
\begin{verbatim}
    :=
\end{verbatim}

\section{Operadores de comparação (boolop)}
\begin{verbatim}
    >  <  >=  <=  == !=
\end{verbatim}

\section{Operadores lógicos (boolop)}
\begin{verbatim}
    && || !
\end{verbatim}

\section{Literais booleanos}
\begin{Verbatim}[commandchars=\\\{\}]
    true false
\end{Verbatim}

\newpage
\chapter{Regras sobre símbolos terminais}
Meta-operadores para definir as expressões regulares:
\begin{verbatim}
    [ ] : para enumerações associadas à -
    *   : repetição
    +   : para repetições de um vez ou mais
    ?   : de zero a uma vez
    .   : como um caractere ``joker'' exceto \n
    ^   : para o complementar de um  [ ]
    |   : para representar uma alternativa
\end{verbatim}

\section{id}
Identificadores podem começar apenas com letras, podem ter números e underline (``\_'') em sua estrutura.
A expressão regular que gera um identificador correto é:
 \begin{verbatim}
   id :  [a-zA-Z][a-zA-Z0-9_]*
\end{verbatim}

\section{label}
 \begin{verbatim}
   label : "@"[a-zA-Z0-9_]*
\end{verbatim}

\section{char}
\begin{verbatim}
   charliteral : \`[^']*\'
\end{verbatim}

\section{string}
\begin{verbatim}
   stringliteral : \"[^"\n]*\"
\end{verbatim}

\section{int}
\begin{verbatim}
   intliteral :  ([0-9]+)
\end{verbatim}

\section{bool}
\begin{verbatim}
   boolliteral :  "true"|"false"
\end{verbatim}

\section{real}
\begin{verbatim}
   exponent : ([E|e]({DIGIT}+))
   real : ([0-9]*[.])?[0-9]+
   realexponent : ([0-9]*[.])?[0-9]+{exponent}?
   realliteral :  ({real}|{realexponent})
\end{verbatim}

\section{Precedência}
A ordem de precedência deve valer para os seguintes operadores (), [], \{\}, $*$, $/$, \%, !, +, -, $<$, $<=$, $>$, $>=$, $:=$, $=$, $==$, $!=$, $\&\&$, $||$. A ordem de precedência pode ser visualizada na Tabela \ref{table:ordem_precedencia}.

\begin{table*}[h]
\renewcommand{\arraystretch}{1.34}
\centering
\begin{tabular}{| c | c | c | c | c | c |}
\hline
\bfseries Operador & \bfseries Precedência  \\
\hline
() & maior \\ \hline
[] &  \\ \hline
\{\} &  \\ \hline
 *, /, \% & \\ \hline
  ! & \\ \hline
 +, - & \\ \hline
$<$, $<=$, $>$, $>=$ & \\ \hline
$==$, $!=$ & \\ \hline
\&\& & \\ \hline
$||$ & \\ \hline
$:=$, = & menor\\ \hline
\end{tabular}
\caption{Ordem de precedência para os operadores.}
\label{table:ordem_precedencia}
\end{table*}

% Pascal tem funções matemáticas na própria linguagem (vamos abortar ou trataremos como se fosse de uma biblioteca?)
\section{Regra para comentários}
    Comentários iniciam por ``\#'' e são eliminados no pré-processamento.
\begin{verbatim}
   linecomment : "#"((.)*)\n
\end{verbatim}


\newpage
\chapter{Analisador Sintático (Parser)}
\section{Conjuntos Predict}
Cada regra sintática tem associada a si um conjunto chamado \textit{predict}, que permite sabermos qual regra deverá ser usada. O conjunto predict basicamente nos diz quais símbolos terminais podem ser o próximo a ser comido se usarmos aquela regra; ou seja, se 

\section{Parser recursivo}
No parser recursivo, é utilizada a pilha de execução do programa 

\subsection{Recuperação de erro}

\section{Parser por tabela}

\subsection{Recuperação de erro}

\newpage
\chapter{Forma de uso}
O código se encontra em \url{https://github.com/raquel-oliveira/PIE}
\section{Compilação}
\begin{verbatim}
    make
\end{verbatim}

\section{Execução}
\subsection{Parser recursivo}
\begin{verbatim}
    ./recursive pathtofile.pie
\end{verbatim}

\subsection{Parser por tabela}
\begin{verbatim}
    ./table pathtofile.pie
\end{verbatim}

\section{Exemplo}
\begin{verbatim}
    make
    ./recursive codesamples/merge_sort.pie
\end{verbatim}

\newpage
\chapter{Exemplos de programas}
%http://rextester.com/GHRH16649
\section{MergeSort}
\begin{footnotesize}
\lstinputlisting{../codesamples/mergesort.pie}
\end{footnotesize}
\newpage
%http://sandbox.mc.edu/~bennet/cs404/doc/qsort_pas.html
\section{Quicksort}
\begin{footnotesize}
\lstinputlisting{../codesamples/quicksort.pie}
\end{footnotesize}
\newpage
%http://rextester.com/UXP1971
\section{Fatorial}
\begin{footnotesize}
\lstinputlisting{../codesamples/factorial.pie}
\end{footnotesize}
\newpage
\section{Program to test grammar}\label{seq:all}
%\begin{footnotesize}
%	\lstinputlisting{../codesamples/all.pie}
%\end{footnotesize}
\verbatiminput{../codesamples/all.pie}

\newpage
\chapter{Participação}
\begin{table}[h]
	\centering
	\label{my-label}
	\begin{tabular}{|l|l|r|}
		\hline
		\textbf{Matrícula} & \textbf{Aluno(a)} & \textbf{Participação (\%)} \\ \hline
		20180008183        & BRENO MAURICIO DE FREITAS VIANA & 22\%                       \\ \hline
		20170153995        & FERNANDA MENEZES PAES ISABEL    & 22\%                       \\ \hline
		2012912517         & LUIZ ARTHUR DE LIMA FREIRE      & 18\%                       \\ \hline
		20180008281        & RAQUEL LOPES DE OLIVEIRA        & 20\%                       \\ \hline
		20180008316        & VÍTOR DE GODEIRO MARQUES        & 0 to fora\%                       \\ \hline
		\multicolumn{2}{|l|}{\textbf{TOTAL}}            & 100\%                      \\ \hline
	\end{tabular}
\end{table}

\nocite{*}
\bibliographystyle{ieeetr}
\bibliography{bibpie}


\end{document}

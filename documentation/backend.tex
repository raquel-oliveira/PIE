\section{Código gerado}
Criamos um back-end do compilador para a geração do código. 
A linguagem usada na geração do código é uma restrição da linguagem C que permite usar apenas os seguintes elementos:
\begin{itemize}
	\item Expressões aritméticas, lógicas, relacionais e chamadas de funções.
	\item Todos os tipos de dados da linguagem.
	\item Todas as declarações da linguagem.
	\item Apenas os seguintes comandos de C:
	\begin{itemize}
		\item Abertura e fechamento de blocos;
		\item Sequenciamento de comandos;
		\item Atribuição;
		\item Chamadas de funções/procedimentos (incluindo de entrada e saída);
		\item rótulos (labels) e comando goto;
		\item Comandos return, break e exit;
		\item Comandos de seleção APENAS da forma:
		\begin{itemize}
			\item \begin{verbatim}
			if( condição ) goto l;
			\end{verbatim}
			\item \begin{verbatim}
		switch ( expressão ) {
		case valor : { … }
		…
		}
		\end{verbatim}
		\item Nenhum outro comando da linguagem deve ser usado, isto inclui comandos de iteração ou de
		seleção estruturados	
		\end{itemize}
	\end{itemize}
\end{itemize}

Ao compilar um arquivo $.pie$ (como descrito em \ref{ch:uso}), um código intermediário é criado na pasta $generated_code$ com o mesmo nome do arquivo de input, porém com a extensão $.c$. Esse arquivo pode ser compilado através do compilador $gcc$, por exemplo:
\begin{verbatim}
gcc <nomedoarquivo>.c -o <nomedoexecutavel>
\end{verbatim}

\section{Estratégias}
Para a implementação da gramática de atributos, usamos a técnica/esquema de tradução dirigida pela sintaxe. Devido a este esquema, notamos a necessidade alterar algumas regras sintáticas (descritas em \ref{ch:sintatica}) adicionando um novo não-terminal logo antes de uma regra caso ela necessite obter uma informação herdada que ela não teria acesso por padrão (no caso da abordagem \textit{bottom-up}). Isso só foi necessário no caso de não terminais que precisavam herdar informações de símbolos em posições diferentes em regras diferentes. Como por exemplo no caso do não terminal $block$ (usados nas regras sintáticas \ref{procedure} e \ref{function}  ):

\begin{lstlisting}[frame=single, language=pie]
<procedure> ::= proc id `(' <param> `)' `;' <decl> <block>
\end{lstlisting}

\begin{lstlisting}[frame=single, language=pie]
<function> ::= func <types> id `(' <param> `)' `;' <decl> <block>
\end{lstlisting}
 Criamos dois símbolos não terminais, G e H, de forma que:
 \begin{lstlisting}[frame=single, language=pie]
 <procedure> ::= proc id `(' <param> `)' `;' <decl> G <block>
 \end{lstlisting}
 
 \begin{lstlisting}[frame=single, language=pie, basicstyle=\small ]
 <function> ::= func <types> id `(' <param> `)' `;' <decl> H <block>
  \end{lstlisting}
  
  Na implementação da gramática de atributos temos então:
 
 \begin{verbatim}
 procedure : {
 			$<attrs>$.sti = $<attrs>0.sti; 
 }
 PROC_TOKEN ID_TOKEN '(' param ')' ';' decl G block {
 		$$.cs = "void " + std::string($3) +"("+ $5.cs + ") {\n" + $8.cs + $10.cs +"\n}\n";
 };
  \end{verbatim}
  \begin{verbatim}
 G : {
		  $<attrs>$.sti = st_union($<attrs>-3.sts,
		                  st_union($<attrs>0.sts, $<attrs>-7.sti)); 
  };
  \end{verbatim}
  \begin{verbatim}
 function : {
  $<attrs>$.sti = $<attrs>0.sti; 
  }
  FUNC_TOKEN types ID_TOKEN '(' param ')' ';' decl H block {
  $$.cs = $3.type + " " + $4 + "(" + $6.cs + ") {\n" + $9.cs + $11.cs + "\n}\n"; 
  } ;
  \end{verbatim}
  \begin{verbatim}
 H : { 
 $<attrs>$.sti = st_union($<attrs>-3.sts, st_union($<attrs>0.sts, $<attrs>-8.sti)); 
 };
 \end{verbatim}
pois no caso do $procedure$ a informação que o $block$ precisa está na posição -7 e no caso da $function$ a informação que o $block$ precisa está na posição -8.

\section{Dificuldades}